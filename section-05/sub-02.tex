\subsection{Relativity of Openness and Closedness}
\begin{majorEx}%5.D
The unique open set in $\RR^1$ which is also open in $\RR^2$ is $\emptyset$.
\end{majorEx}
\begin{proof}
In order for a set $X \in \RR^2$ to be open, it must be the union of open balls. Any open ball in $\RR^2$ will contain some values that do not fit the form $(x, 0)$ for some $x \in \RR$, so it is not possible for a union of open balls to be exlusively values in $\RR$. $\emptyset$ is trivially open in both $\RR$ and $\RR^2$, so it is the unique set that is open in both. 
\end{proof}

Primary author: Willie Kaufman

\begin{majorEx}%5.E
An open set of an open subspace is open in the ambient space, i.e., if $A \in \Omega$, then $\Omega_A \in \Omega$.
\end{majorEx}
\begin{proof}
  Let $A\in \Omega$, and $B \in \Omega_A$ be arbitrary. We see by
  definition of 
  $\Omega_A$, that $B = A \cap V$ for some $V\in \Omega$. As $A\in
  \Omega$, and topological spaces are closed under finite
  intersection, we have that $B \in \Omega$. As $A$,$B$ was arbitrary, we
  see that this is true for any open subspace.
\end{proof}
Primary Author: Reilly Noonan Grant
\begin{majorEx}%5.F
Closed sets of a closed subspace are closed in the ambient space.
\end{majorEx}
\begin{proof}
  Let $A$ be an arbitrary closed subspace, and $B$ be an arbitrary
  closed set in that subspace. We see that for some $V\in \Omega$,
  that $B= A \setminus (A\cap V)$ where $V \in \Omega$. As $V$ is
  open, we see that $X \setminus V$ is closed, and as closed sets are
  closed under intersection, we see that $A \cap (X \setminus V)$ is
  also closed. We will now show that $B =A \cap (X \setminus V)$. Let
  $b\in B$ be arbitrary. We see that $b\in A$, but that $b \notin
  A \cap V$, and thus $b \notin V$. We now can see that as $A\subset
  X$, that $b\in X$, and thus that $b\in X$ and as $b\notin V$ $b\in
  (X\setminus V)$. As $b \in A$, we have that $b\in A \cap (X\setminus
  V)$. As $b$ was arbitrary, we have that $B \subset A \cap (X \setminus V)$.
  We now let  $a\in A \cap (X \setminus V)$ be arbitrary. By
  definition of intersection, we have that $a \in A$, and 
  $a \in  (X \setminus V)$. By the definition of the difference of
  sets, we have that $a \notin V$, and thus  $a \notin (A\cap V)$. We
  now can see that $a \in A \setminus (A\cap V) = B$, and thus as $a$
  was arbitrary, and $B \subset A \cap (X \setminus V)$, we have that
  $B \subset A \cap (X \setminus V)$, and thus that $B$ is closed. As
  $B$ and $A$ were arbitrary, we have that closed sets of a closed
  subspace are closed in the ambient space. 
\end{proof}
Primary Author: Reilly Noonan Grant

\begin{minorEx}%5.5
Prove that a set $U$ is open in $X$ iff each point in $U$ has a neighborhood $V$ in $X$ such that $U \cap V$ is open in $V$.
\end{minorEx}

\begin{proof}
  We will first show that if a set $U$ is open in $X$, that every
  point in $U$ has a neighborhood $V$ in $X$ such that $U \cap V$ is
  open in $V$. Let $x \in U$ be arbitrary. As $U$is open, we know that
  $U$ is a neighborhood of $x$, and by definition of a subspace, we
  know that $U\cap U=U$ is open in $U$.

  We will now show that if every point in $U$ has a neighborhood $V$
  in $X$ such that $U \cap V$ is open in $V$, that $U$ is open in
  $X$. Let $x \in U$ be arbitrary. We see that $U \cap V \subset
  U$. We also see that as $V$ is a neighborhood, that $V$ is open, and
  thus as $U \cap V $ is open in $V$, by 5.E, we have that $U \cap V$
  is open in $X$. As $x$ was arbitrary, we have that every point in
  $U$ is contained within an open set contained within $U$. We thus
  see that if we take the union of all these set $A$, that $U\subset
  A$, as $A$ contains all points in $U$, and $A \subset U$ as every
  set is contained within $U$. We thus have that $U=A$, and as $A$ is
  an arbitrary collection of open sets, we have that $A$ and thus $U$
  is open.
\end{proof}

Primary Author: Reilly Noonan Grant

\begin{minorEx}%5.6
Show that the property of being closed is not local.
\end{minorEx}

\begin{proof}
    We will show that there is a closed set $F$ such that for every open
    neighborhood $V$ around $x \in F$ the set $F \cap V$ is not closed with
    respect to $V$.

    In particular, if $F=[0,1]$, then any neighborhood containing $V$ can be
    written as a union of open intervals; it suffices to show that any open
    interval containing $0$ is fails. Let $a < 0 < b < 1$ and consider $(a,b)$.
    Then $(a,b) \cap [0, 1] = [0, b)$. But $[0, b)$ is not closed, since its
        complement is not open. Hence, for any open interval $V$ containing 0,
        $F \cap V$ is not closed.
\end{proof}

Primary author: David Kraemer

\begin{majorEx}%5.G
Let $(X, \Omega)$ be a topological space, $X \supset A \supset B$. Then $(\Omega_A)_B=\Omega_B$, i.e., the topology induced on $B$ by the relative topology of $A$ coincides with the topology induced on $B$ directly from $X$.
\end{majorEx}


\begin{proof}
  We will show $(\Omega_A)_B=\Omega_B$ by double containment. We first
  let $C \in (\Omega_A)_B$ be arbitrary. We see that by the for some
  $V \in \Omega_A$, that $C = B \cap V$. We also see that as $V \in
  \Omega_A$, that $V = A \cap  V^\prime$, and thus $C = B \cap A \cap
  V^\prime$. We see as $B\subset A$, that 
  $B \cap A \cap V^\prime=B \cap V^\prime$, and thus $C =B \cap
  V^\prime$. As $V^\prime \in \Omega$, we know that $C \in \Omega_B$
  by definition. As $C$ was arbitrary, we have that 
  $(\Omega_A)_B \subset \Omega_B$.

  We now let $D \in \Omega_B$ be arbitrary. We see that for some $V
  \in \Omega$, $D = B \cap V$. As $B\subset A$, we have that $B \cap V
  = A\cap B \cap V=A\cap (B \cap V)$. We now can see by definition of
  $\Omega_A$ that $(B \cap V) \in \Omega_A$. We now see by the
  definition of $(\Omega_A)_B$ that $D\in (\Omega_A)_B$. As $D$ was
  arbitrary, and $(\Omega_A)_B \subset \Omega_B$, we have 
  $(\Omega_A)_B = \Omega_B$, and thus the topology induced on $B$ by
  the relative topology of $A$ coincides with the topology induced on
  $B$ directly from $X$.  
\end{proof}
Primary Author: Reilly Noonan Grant

\begin{minorEx}%5.7
Let $(X, \rho)$ be a metric space, $A \subset X$. Then the topology on $A$ generated by the induced metric $\rho_{|AxA}$ coincides with the relative topology induced on $A$ by the metric topology on $X$.
\end{minorEx}
\begin{proof}
The topology on $A$ generated by the induced metric $\rho_{|AxA}$ is equivalent to the portion of each open ball contained in $A$. Let $\Omega$ equal the set of all open balls in $X$. The relative topology induced on $A$ by the metric topology on $X$ is equivalent to $\Omega_A = {\omega \cap A | \omega \in \Omega}$. Given a portion of a ball centered at $b$ with radius $r$ in the topology generated by the induced metric, that same ball - as defined by its center and radius - in $\Omega$ intersected with $A$ will be the same portion of ball. Similarly, given a ball in $\Omega$ intersected with $A$ with radius $b_2$ and radius $r_2$, that same ball - as defined by its center and radius - in the topology generated by the induced metric will be equivalent. Since any element in either topology has a mirror in the other, the topologies are equivalent.

Primary author: Willie Kaufman


\begin{minorEx}[Riddle]%5.8
The statement in 5.7 is equivalent to a pair of inclusions. Which of these is less obvious?
\end{minorEx}
\begin{proof)[Answer]
5.7 claims that the portion of open balls contained in $A \subseteq {\omega cap \A | \omega is an open ball in X}$ and that ${\omega cap \A | \omega is an open ball in X} \subseteq$ the portion of open balls contained in A. The first is fairly intuitive; any portion of open balls contained in $A$ is definitely contained in a really big open ball surrounding $A$ intersected with $A$. The second is less so, mostly because my intuition for what a metric that is restricted to a certain space looks like.

Primary author: Willie Kaufman

