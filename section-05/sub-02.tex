\subsection{Relativity of Openness and Closedness}
\begin{majorEx}%5.D
The unique open set in $\RR^1$ which is also open in $\RR^2$ is $\emptyset$.
\end{majorEx}
\begin{proof}
\end{proof}

\begin{majorEx}%5.E
An open set of an open subspace is open in the ambient space, i.e., if $A \in \Omega$, then $\Omega_A \in \Omega$.
\end{majorEx}
\begin{proof}
\end{proof}

\begin{majorEx}%5.F
Closed sets of a closed subspace are closed in the ambient space.
\end{majorEx}
\begin{proof}
\end{proof}

\begin{minorEx}%5.5
Prove that a set $U$ is open in $X$ iff each point in $U$ has a neighborhood $V$ in $X$ such that $U \cap B$ is open in $V$.
\end{minorEx}

\begin{minorEx}%5.6
Show that the property of being closed is not local.
\end{minorEx}

\begin{majorEx}%5.G
Let $(X, \Omega)$ be a topological space, $X \supset A \supset B$. Then $(\Omega_A)_B=\Omega_B$, i.e., the topology induced on $B$ by the relative topology of $A$ coincides with the topology induced on $B$ directly from $X$.
\end{majorEx}

\begin{minorEx}%5.7
Let $(X, \rho)$ be a metric space, $A \subset X$. Then the topology on $A$ generated by the induced metric $\rho_AxA$ coincides with the relative topology induced on $A$ by the metric topology on $X$.
\end{minorEx}

\begin{minorEx}[Riddle]%5.8
The statement in 5.7 is equivalent to a pair of inclusions. Which of these is less obvious?
\end{minorEx}


