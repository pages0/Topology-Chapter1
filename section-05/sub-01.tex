\subsection{Topology for a Subset of a Space}

\begin{majorEx}%5.A
The collection $\Omega_A$ is a topological structure in $A$.
\end{majorEx}
\begin{proof}
  We first see that as $X\in \Omega$, that $A\cap X \in \Omega_A$ and
  thus $A\in \Omega_A$. We also have that as $\emptyset \in \Omega$
  that $A\cap \emptyset \in \Omega_A$, and thus $\emptyset \in
  \Omega_A$. As $A,\emptyset \in \Omega_A$, we have that Axiom 3 of a
  topological space is satisfied.

  We will now show that the union of any collection of sets that are
  elements of $\Omega_A$ belong to $\Omega_A$. Let $B$ be an arbitrary
  union of elements of $\Omega_A$. We also let $V\in \Omega$ be defined to be
  the arbitrary union of all sets $V^\prime$ such that 
  $A\cap V^\prime \subset B$. As $\Omega$ is a topology, we know that
  it is closed under arbitrary unions, and thus that $V\in \Omega$. We
  thus have by the definition of $\Omega_A$ that $A\cap V\in
  \Omega_A$. We will now show that $B =A\cap V$, 
  and thus that $B\in \Omega_A$.
  Let $b\in B$ be arbitrary.
  We see that by the definition of $B$, that $b\in A$,
  and for some $V^\prime\subset A$, $b\in V^\prime$.
  We thus have by the definition of $V$, that $b \in V$, and as $b\in
  A$, we have that $b \in A \cap V$. As $b$ was arbitrary, we have
  that $B \subset A \cap V$. We now let $a \in A \cap V$ be
  arbitrary. We see that $a\in A$, and by the definition of $V$, for
  some $A\cap V^\prime\subset B$, $a \in A\cap V^\prime$. We thus have
  that $a \in B$, and as $a$ was arbitrary, and $B \subset A \cap V$,
  we have that $B = A \cap V$. As $A \cap V$ is in $\Omega_A$, and $B$
  was an arbitrary union of elements of  $\Omega_A$, we see that
  $\Omega_A$ is closed under arbitrary union.

  We will now show that $\Omega_A$ is closed under finite
  intersection. Let $B$ be an arbitrary finite intersection of sets in
  $\Omega_A$. We also let $V$ be defined such that $V$ is the
  intersection of every $V^\prime$ such that $A\cap V^\prime$ is an
  element of the collection of sets whose finite intersection defines
  $B$. As $\Omega$ is a topological space, we know that $V\in \Omega$,
  as $\Omega$ is closed under finite intersection. We thus have that
  $A\cap V \in \Omega_A$, we will now show that $B =A\cap V$, and thus
  that $B\in \Omega_A$. Let $b \in B$ be arbitrary. We see that by the
  definition of intersection, that $b \in A$, and that for every
  $V^\prime$ such that $A\cap V^\prime$ the collection of sets whose
  finite intersection defines $B$, we have that $a \in V^\prime$. We
  thus have that by the definition of $V$, $b\in V$, and thus $b \in A
  \cap V$. As $b$ was arbitarary, we know that $B \subset A \cap
  V$. We now let $a \in A \cap V$ be arbitrary. We see that for any $A
  \cap V^\prime$ which defines $B$, that $a \in A$, and by definition
  of $V$, $a \in V^\prime$. We thus have that by definition of
  intersection, that $a \in B$, and as $a$ was arbitrary, and 
  $B \subset A \cap V$, we have that $B = A \cap V$.
\end{proof}

Primary Author: Reilly Noonan Grant

\begin{majorEx}%5.B
The canonical topology on $\RR^1$ coincides with the topology induced on $\RR^1$ as a subspace of $\RR^2$.
\end{majorEx}
\begin{proof}
Let $\Omega$ be the canonical topology on $\RR^1$. Let $\Sigma$ be the canonical topology on $\RR^2$
and let $\sigma \in \Sigma$ be arbitrary. Let $R$ be the set ${x | x \in \RR^2 and x = (R, 0) for some R \in \RR}$. Consider the set $S = {R \cap \sigma}$. Each ball that forms $\sigma$ intersects $R$ for some open interval, so $S$ will be the union of open intervals, and so will be in the canonical topology on $\RR^1$. We can also reverse engineer a $\sigma$ that generates an arbitrary $\omega \in \Omega$ by choosing an appropriate ball for each open interval, the union of which form $\omega$, so there are no elements $\omega \in \Omega$ that are not induced by considering $\RR^1$ as a subspace of $\RR^2$, and so the sets $\Omega$ and $\Sigma_R$ are equivalent.
\end{proof}

Primary author: Willie Kaufman

\begin{minorEx}[Riddle]%5.1
How to construct a base for at topology induced on $A$ by using a base for the topology on $X$.
\end{minorEx}
If $\Sigma$ is a base for the topology on $X$, the set ${A \cap \sigma | \sigma \in \Sigma}$ forms a base for $A$. 

Primary author: Willie Kaufman

\begin{minorEx}%5.2
Describe the topological structures induced
(1) on the set $\NN$ of positive integers by the topology of the real line:
(2) on $\NN$ by the topology of the arrow
(3) on the two-element set {1, 2} by the topology of $\RR_T$
(4) on the same set by the topology of the arrow
\end{minorEx}

\begin{proof}[Answer]
  (1) We see that for any $n\in \NN$, that $(n-1/2, n+1/2)$ is in the
  topology of the real line, and thus that $\{n\}$ is in the induced
  topology, and thus as $n$ was arbitrary every element in $\NN$ would
  have in a set in the topology which contains only it, and thus this
  topology would be the discrete topology on $\NN$.

  (2) This topology would be the collection of all set in $\NN$ such
  of the form $[n, \infty)$ where $n$ is a natural number greater than
  $0$. We can see this, by seeing that for any $n>0$, $(n-1/2,\infty)$
  is in the topology of the real line, and that if $n$ is in a set, by
  the definition of the arrow, all numbers greater than it must be in
  the set.
  
  (3) We see that this would be the discrete topology, as
  $\emptyset,\{1\},\{2\}, \{1,2\}$ could all individually be included
    by considering the finite complement of $1$, and $2$, $2$, $1$,
    and $0$ respectively.

  (4) We see that this would be the topology $\emptyset,\{2\},
  \{1,2\}$, as if $1$ is in the set, any set in the topology of the
  arrow would also have $2$ in it.
\end{proof}
Primary Author: Reilly Noonan Grant

\begin{minorEx}%5.3
Is the half open interval $[0, 1)$ open in the segment $[0, 2]$, regarded as a subspace of the real line?
\end{minorEx}
\begin{proof}[Answer]
For $[0, 1)$ to be open in $[0, 2]$, it must be the case that there is an arbitrary union of open sets that, when intersected with $[0, 2]$, yields $[0, 1)$. If we choose the open interval $(-1, 1)$ then intersect it with $[0, 2]$ to get an element of our relative topology, we get $[0, 1)$, i.e. $[0, 1)$ is an empty set in this relative topology.

Primary author: Willie Kaufman

\begin{majorEx}%5.C
A set $F$ is closed in a subspace $A \subset X$ iff $F$ is the intersection of $A$ and a closed subset of $X$.
\end{majorEx}
\begin{proof}
\end{proof}

\begin{minorEx}%5.4
If a subset of a subspace is open (respectively, closed) in the ambient space, then it also open (respectively, closed) in the subspace.
\end{minorEx}

\begin{proof}
  Let $B$ be an arbitrary set in a subspace $(A,\Omega_A)$ of the
  space $(X,\Omega)$. We will show that $B$is an open in $(X,\Omega)$,
  that it is open in $(A,\Omega_A)$. We first see that as $B$ is in
  $(A,\Omega_A)$, that $B \subset A$. As $B$ is open, we know that
  $B\in \Omega$, and as $B \subset A$, we know that $A\cap B=B$. We
  now see by the definition of of subspace, that $A\cap B \in
  \Omega_A$, and thus is an open set, and thus $B$ is an open set.
\end{proof}

Primary Author: Reilly Noonan Grant

