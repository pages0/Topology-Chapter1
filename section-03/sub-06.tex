\subsection{Hierarchy of Topologies}
\begin{majorEx}
3.D. Riddle. Formulate a necessary and sufficient condition for two bases
to be equivalent without explicitly mentioning the topological structures
determined by the bases.
\end{majorEx}
In order for two bases to describe the same topology, it must be that if an open set $O$ is the union of sets in $B_1$, it is also equal to the union of sets in $B_2$. WLOG, assume a set $A$ is equal to the union of some sets in $B_1$. For an arbitrary element $x \in A$, we must have a set $b \in B_2$ such that $b \subset A$ and $x \in b$ for an arbitrary $x$. In order to guarantee that this is the case without referring to $A$, for each set $b \in B_1$ and element $x \in b$, there exists a set $c \in B_2$ such that $c \subset b$ and $x \in c$, as well as this being the case in the other direction.
