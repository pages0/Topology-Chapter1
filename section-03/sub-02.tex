\subsection{When a Collection of Sets is a Base}
\begin{majorEx}
  A collection $\Sigma$ of open sets is a base for the topology iff for
  every open set $U$ and every point $x\in U$  
there is a set $V \in
  \Sigma$ such that $x \in V \subset U$
\end{majorEx}
\begin{proof}
We will first show that if $\Sigma$ is a base for the  topology if for
every open set $U$ and every point $x \in U$ there is a set $V \in
\Sigma$ such that $x \in V \subset U$. Let $U\in \Sigma$ be arbitrary
and let $x\in U$ be arbitrary. By the definition of a basis, we know
that because $U$ is an open set that $U$ is a union of sets in $\Sigma
$. We know because $x \in U$ that for at least one of the sets in this
union, $x$ will be in the set. We will call one of these sets $V$.
We see that if $V$ had an
element which was not in $U$, that the union of sets would not be
equal to $U$, and as $x\in V$, we know that $V\subset U$. We thus have
that because $U$ and $x$ were arbitrary, that the statement holds true
in all cases.

We will now show that if for every open set $U$ and every point $x\in U$  
there is a set $V \in \Sigma$ such that $x \in V \subset U$ that
$\Sigma$ is a base for the topology. Let $U$ be an arbitrary open
set. We define $A$ to be the union of all $V$ that correspond to some
point in $U$. We see that as for each $V$, $V \subset U$, we know that
$A \subset U$, as if any element of $A$ were not in $U$, then for some
set $V$, $V$ would not be a subset of $U$. We also see that $U\subset
A$ as for each $x\in U$, we have that $x\in V$ for some $V$, and thus
$x\in A$ by the definition of a union. We thus have that by 1.E that
$A= U$, and as $U$ was arbitrary, we know that any $U$ can is a union
of sets in $\Sigma$, and thus $\Sigma$ is a base for the topology.
\end{proof}

\begin{majorEx}%3.B
A collection $\Sigma$ of subsets of a set $X$ is a base for a certain topology on $X$ iff $X$ is the union of all sets in $\Sigma$ and
the intersection of any two sets in $\Sigma$ is the union of some sets in $\Sigma$.
\end{majorEx}
First we will prove that if $\Sigma$ is the base for a topology on $X$, $X = \bigcup_(s \in \Sigma) s$ and the intersection of any two sets in $\Sigma$ is the union of some sets in $\Sigma$. First, let $\Omega$ be the topology for which $\Sigma$ is a base. We have that $X \in \Omega$ and, by the definition of base, $X$ must the union of some sets in $\Sigma$. Since every element of each set in $\Sigma$ must be in $X$, $X$ being the union of some sets in $\Sigma$ implies it is the union of all sets in $\Sigma$. Now let $s_1$ and $s_2$ be two sets in $\Sigma$. Because the sets are in $\Sigma$, they are open, and so their intersection is also open. Now we know that their intersection is equal to the union of some open sets in $\Omega$, and since all open sets in $\Omega$ are equal to the union of some open sets in $\Sigma$ by the definition of base, we can replace those open sets with the elements of $\Sigma$ whose union equals them and the intersection of $s_1$ and $s_2$ must be the union of some sets in $\Sigma$.
Now we prove that if $X = \bigcup_(s \in \Sigma) s$ and the intersection of any two sets in $\Sigma$ is the union of some sets in $\Sigma$, $\Sigma$ is a base for some topology on $X$. Let $\Omega$ be the set of arbitrary unions of elements in $\Sigma$ and \emptyset. We will show that $\Omega$ is a topology.
First, it is the case that $X$ and $\emptyset \in \Omega$, by the way we defined $\Omega$ and the assumptions we started with. So the third axiom of topology is satisfied.
Second, consider the finite intersection of sets in $\Omega$. Let $u_1, u_2, ..., u_n \in \Omega$ to be arbitrary. First, notice that the intersection of $u_1$ and $u_2$ is the union of some sets in $\Sigma$, i.e. is in $\Omega$. Let $u_1 = \bigcup_(i \in I) \sigma_i$, $u_2= \bigcup_(j \in J) \sigma_j$, etc. Then their intersection is equal to $\bigcup_(i \in I, j \in J) (\sigma_i \cap \sigma_j)$. By our assumptions, each $(\sigma_i \cap \sigma_j)$ is the union of some sets in $\Sigma$, i.e. is in $\Omega$. We can apply intersections in any order, so the fact that the intersection of two arbitrary sets in $\Omega$ is in $\Omega$ assures us that, for any finite number of sets in $\Omega$, their intersections are also in $\Omega$, and the second axiom of topology is satisfied.
Third, consider the arbitrary union of elements in $\Omega$. Let $u_1, u_2, ... \in \Omega$ be arbitrary. The properties of unions assure us that the union of all $u_i$ is the union of each element in $\Sigma$ such that there exists $u_i$ where $u_i$ was formed by a union including that element of $\Sigma$. Then we have that the union of $u_1, u_2, ...$ is the union of some number of sets in $\Sigma$, i.e. is in $\Omega$, and the first axiom of topology is satisfied.

\begin{majorEx}%3.C
Show that the second condition in 3.B (on the intersection) is equivalent to the following one: the intersection of any two sets in $\Sigma$ contains, together with any its points, a certain set in $\Sigma$ containing this point (cf. Theorem 3A).
\end{majorEx}
Let $\Sigma$ be a collection of subsets of a set $X$ such that the intersection of any two sets in $\Sigma$ is the union of some sets in $\Sigma$. Let $A, B \in \Sigma$ be arbitrary. If $c \in A \cap B$, it must be the case that there exists some set in $\Sigma$ contained in $A \cap B$ that contains $c$, because otherwise we will be unable to find a collection of sets in $\Sigma$ whose union equals $A \cap B$; we would have no way to include $c$ in this union without some set that "overflows".
Now we consider the other direction of the equivalence; let $\Sigma$ be a collection of subsets of a set $X$ such that, for any $A, B \in \Sigma$ and $c \in A \cap B$, there exists $S \in \Sigma$ such that $c \in S$ and $S \subseteq A \cap B$. Consider the set $\Alpha = {S | for some c \in A \cap B, S is a set that satisfies the previous assumption}$. Consider $\bigcup_(\alpha \in \Alpha) \alpha$. For each points $c \in A \cap B$, $c$ is in this union, and because each $\alpha \subseteq A \cap B$, $\bipcup_(\alpha \in \Alpha) \alpha$ = $A \cap B$ by double containment. Each $\alpha \in \Alpha$ is in $\Sigma$ by the way we chose them, and so for an arbitrary $A, B \in \Sigma$, their intersection is the union of some sets - those in $\Alpha$ - in $\Sigma$.
