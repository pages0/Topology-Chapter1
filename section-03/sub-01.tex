\subsection{Definition of Base}
\begin{minorEx} %3.1
Can two distinct topological structures have the same base?
\end{minorEx}

We claim that two distinct topological structures must have different bases.

\begin{proof}
Assume that two different topological structure $\Omega_1$ and $\Omega_2$ have the same base $S_b$. Every subset in $\Omega_1$ can be expressed as unions of elements in $S_b$ which will be in $\Omega_2$ by definition of topological structure. So we reach a conclusion that $\Omega_1$ is the same as $\Omega_2$ since they contain the same elements, a contradiction.
\end{proof}

\begin{minorEx}
    Find some bases for the topology of
    \begin{enumerate}
        \item a discrete space;
        \item the weird space;
        \item an indiscrete space;
        \item the arrow.
    \end{enumerate}
    Try to choose the smallest base possible.
\end{minorEx}

\begin{proof}[Answer]
    \begin{enumerate}
        \item Let $\Sigma = \set{\set{x} : x \in X} \cup \set{\emptyset}$ be the set of all
            singleton subsets of $X$. Then any nonempty subset of $X$ can be
            represented by the union of all its singleton element sets.
        \item Let $\Sigma = \set{\set{a}, \set{b}, \set{c}, \set{d}}$. Every
            element of $\Omega$ is a union of some of these singletons.
        \item Let $\Sigma = \set{X, \emptyset}$. This is the smallest basis,
            since both are needed to generate $\set{X, \emptyset} = \Omega$.
        \item Whenever $a < b$, we have $(a, \infty) \cup (b, \infty) = (a,
            \infty)$, so each open set of the arrow is necessary to generate
            itself. Hence, $\Sigma = \Omega$.
    \end{enumerate}
\end{proof}

Primary Author: Jimin Tan

\begin{minorEx}[Riddle]%3.3
  Prove that any base of the canonical topology on $\RR$ can be decreased
\end{minorEx}
\begin{proof}
  Let $\Sigma$ be a base of the canonical topology on $\RR$. Let $A$ be an element of 
  $\Sigma$. We see that $A$ is an open set, and thus an arbitrary
  union of open intervals. We let $(a,b)$  be an arbitrary open
  interval such that $(a,b)\cup B =A$ where $B$ is a possibly empty
  union of open intervals. We see that by the definition of a base,
  that as $(a,b)$ is in the cannonical topology, and $B$ is in the
  cannonical topology, that there must already be some collection of
  elements in $\Sigma$ such that $(a,b)$ and $B$ are unions of those
  elements. We thus see that if $A$ is of the form $(a,b)\cup B$ , that
  it can be removed from $\Sigma$, and thus $\Sigma$ 
  can be decreased. 

  We now suppose that $A$ is not a union of open intervals, and is instead
  just an open interval $(a,b)$. We now let $c\in (a,b)$, and
  $d\in (a,c)$, and $e\in (c,b)$. We see that $(a,e)$ and $(d,b)$ are
  in the canonical topology on $\RR$, and thus that they are the union
  of some set of elements in $\Sigma$. This set of elements does not
  include $A$, as $d,e\in A$, and thus a union of $A$ with other
  elements would contain $d$ and $e$. We now see that as 
  $(a,e)\cup (d,b) = (a,b)$, that a union of elements in $\Sigma$ will
  include $A$, and thus that removing $A$ from $\Sigma$ would not
  remove any element from the cannonical topology, and thus that
  $\Sigma$ can be decreased. As $A$ was arbitrary, and $\Sigma$ was
  arbitrary, and in all cases we were able to remove $A$ from
  $\Sigma$, we see that any base of the cannonical topology 
  can be decreased.
\end{proof}

Primary author: Reilly Noonan Grant

\begin{minorEx}[Riddle]%3.4
What topological structures have exactly one base?
\end{minorEx}
The first thing to note is that given a base $\Sigma$ of a topology $\Omega$, if there exists an open set $o$ in $\Omega$ that is not in $\Sigma$, there exists another base $\Sigma \cup o$. SO the only topologies for which there are a single base are those for which the only base is all of $\Omega$. We know the set $X$ must be the union of sets in $\Omega$, so there must be no collection of sets in $\Omega$, the union of which is $X$, except the entire set. This is the case when, for all sets $a \in \Omega$, there does not exist some collection of sets $C$ in $\Omega$ \ $a$ for which $a \subset C$.
