\subsection{Definition of Base}
\begin{minorEx}[Riddle]%3.3
  Prove that any base of the canonical topology on $\RR$ can be decreased
\end{minorEx}
\begin{proof}
  Let $\Sigma$ be a base of the canonical topology on $\RR$. Let $A$ be an element of 
  $\Sigma$. We see that $A$ is an open set, and thus an aritrary union of open intervals. If $A$ is the union of more than one open interval, we can separte these unions, and thus decrease the size of $A$, and we have show our case. We thus assume that $A$ is one open interval, defined by $(a,b)$. We now see that if $c$ is the mean of $a$ and $b$, and $d$ is the mean of $c$ and $a$, and $f$is the mean of $c$ and $b$, that $(a,f)$and 
  $(d,b)$ are both smaller than $(a,b)$, but as $(a,f) \cup (d,b)$, we see that a base with $(a,f)$and 
  $(d,b)$ would have all the same properties as a base with $A$.
\end{proof}

\begin{minorEx}[Riddle]%3.4
What topological structures have exactly one base?
\end{minorEx}
The first thing to note is that given a base $\Sigma$ of a topology $\Omega$, if there exists an open set $o$ in $\Omega$ that is not in $\Sigma$, there exists another base $\Sigma \cup o$. SO the only topologies for which there are a single base are those for which the only base is all of $\Omega$. We know the set $X$ must be the union of sets in $\Omega$, so there must be no collection of sets in $\Omega$, the union of which is $X$, except the entire set. This is the case when, for all sets $a \in \Omega$, there does not exist some collection of sets $C$ in $\Omega$ \ $a$ for which $a \subset C$.
