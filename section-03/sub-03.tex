\subsection{Bases for Plane}

\begin{minorEx}
    Prove that every element of $\Sigma^2$ is a union of elements in
    $\Sigma^\infty$.
\end{minorEx}

\begin{minorEx}
    Prove that the intersection of any two elements of $\Sigma^1$ is a union of
    elements in $\Sigma^1$.
\end{minorEx}

\begin{proof}
    Let $\sigma_1, \sigma_2 \in \Sigma^1$. Then $\sigma_1 \cap \sigma_2$ is a
    rectangle with width $w$ and length $\ell$. Without loss of generality,
    assume $w \leq \ell$. Then if we take $B_w(c_1)$ and $B_w(c_2)$, where $c_i$
    corresponds to the square of length $w$ which coincides with one of the
    edges of $\sigma_1 \cap \sigma_2$ of length $w$, we claim that $B_w(c_1)
    \cup B_w(c_2) = \sigma_1 \cap \sigma_2$.

    If $x \in \sigma_1 \cap \sigma_2$, then 
\end{proof}<++>

\begin{minorEx}%3.7
  Prove that each of the collections $\Sigma^2$,$\Sigma^\infty$, and
  $\Sigma^1$ is a base for some topological structure in $\RR^2$, and
  that the structures determined by these collections coincide
\end{minorEx}
\begin{proof}
  We will first show that $\Sigma^2$ is a base for the topology of all
  shapes in $\RR^2$ which don't include their boundries. Let $U$ be an
  arbitrary element of this topology which is not the emptyset, and
  let $x\in U$ be arbitrary. We now let $c$ be a point not in $U$ such
  that the distance between $x$ and $c$ is as short as the shortest
  distance between $x$ and any other point not in $U$. If no such
  point exists, then $U$ contains every $x\in\RR^2$, and thus $U =
  \RR^2$. We see in this case, that that as $\Sigma^2$ contains all open
  discs, that the disk $V$ centered at $x$ with a radius of $1$ is in
  $\Sigma^2$, and as $V\in \RR^2$, we know $V \in U$, and thus for
  some $V\in \Sigma^2$  $x\in V \subset U$. We now suppose that $c$
  exists. We see that as the distance is the shortest, that the disk
  centered at $x$, $V$, with a radius equal to the distance between $c$ and
  $x$ will be contained within $U$. Because $\Sigma^2$ contains all open
  discs, we see that $V\in\Sigma^2$ and as it is centered at $x$, we
  know that $x \in V$. We thus have that for some $V$, $x\in V \subset
  U$. We now can see that as $x$ and $U$ were arbitrary, that by 3.A,
  we know that $\Sigma^2$ is a base for the topology of all open
  shapes in $\RR^2$. We can also see that if with each $V$ in the
  above aproach we inscribe either an element of $\Sigma^1$ or
  $\Sigma^\infty$ which would also contain $x$, and all be contained
  within the open set. Thus by a similar argument, and 3.A, we know
  that $\Sigma^1$ and $\Sigma^\infty$ are also bases for the topology of all open
  shapes in $\RR^2$.

\end{proof}
