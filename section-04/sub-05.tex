\subsection{Surprising Balls}

\begin{minorEx}%4.7
  What are balls and spheres in $\RR^2$ equipped with the metrics
  of $4.1$ and $4.2$ ?
\end{minorEx}

\begin{proof}[Answer]
Let us first consider a ball $B_r(a)$ with the metric of $4.1$. We see that this
ball will be defined by all the points in which $|x_1-a_1|<r$,
and  $|x_2-a_2|<r$ where $a=(a_1,a_2)$, and an arbitrary point is given
by $x= (x_1,x_2)$. We can see from this that a ball with the metric of
$4.1$ will be a square centered around $a$, with sides parallel to
the x-axis and the y-axis with a height, and width
of $2r$, with the boundry not included. We can similarly see that a
sphere with the metric $4.1$ would be all points in which $|x_1-a_1|=r$,
and  $|x_2-a_2|=r$ , and thus would be the boundry of a square with
the same dimensions as the ball with radius $r$.

We will now consider a ball $B_r(a)$ with the metric of $4.2$. We see
that this ball will be defined by all the points in which 
$|x_1-a_1| + |x_2-a_2|<r$, where $a=(a_1,a_2)$, and
an arbitrary point is given by $x= (x_1,x_2)$. We can see from this
that a ball with the metric of $4.2$ will be the inside of a square centered around
$a$, rotated such that the diagonals are parallel to the x-axis and
the y-axis, and with diagonals of length $2r$. We would also see that
a sphere would have the same dimensions as a ball, but would include
the boundry instead of the inside.
\end{proof}

\begin{minorEx}%4.8
Find $D_1(a)$, $D_(1/2)(a)$, and $S_(1/2)(a)$ in the space of 4.A.
\end{minorEx}
\begin{proof}[Answer]
For a set $X$ with the metric described in 4.A, $D_1(a)$ is the entire set $X$, $D_(1/2)(a)$ is a singleton with just the element $a$, and $S_(1/2)(a)$ is the emptyset.
\end{proof}

\begin{minorEx}%4.9
    Find a metric space and two balls in it such that the ball with the smaller
    radius contains the ball with the bigger one and does not coincide with it.
\end{minorEx}

\begin{proof}[Answer]
  Consider the metric space on the subset of $\RR$, $\{-1,0,1\}$. We
  see that the ball $B_{1.2}(0) = \{-1,0,1}$, and that the ball 
  $B_{1.5}(-1)=\{-1,0\}$. We thus see that $B_{1.5}(-1) \subset
  B_{1.2}(0)$, and $B_{1.5}(-1) \not\subset B_{1.2}(0)$, and thus this
  space has a ball with a smaller radius which contains a ball with a
  larger radius, which does not coincidde with it
  
\end{proof}

\begin{minorEx}
    What is the minimal number of points in the space which is required to be
    constructed in 4.9?
\end{minorEx}

\begin{minorEx}
    Prove that the largest radius in 4.9 is at most twice the smaller radius.
\end{minorEx}
