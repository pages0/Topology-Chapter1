\subsection{FurtherExamples}

\begin{minorEx}%4.1
  Prove that $\RR^n \times \RR^n : \rightarrow \RR_+: (x,y) \mapsto
  max_{i=1,...,n} |x_i-y_i|$ is a metric.
\end{minorEx}

\begin{proof}
  We first see that if $\rho(x,y)=0$, that for some $i$, we know by
  $4.B$ that $x_i = y_i$. As this is the maximum value, and the absolute
  value function is never negative, we thus know that this must be
  true for every $i$, and thus for every $i$ $x_i = y_i$. We thus have
  that $\rho(x,y)=0$ if and only if $x=y$.

  We will now show that for any $x,y\in \RR^n$ that 
  $\rho(x,y) = \rho(y,x)$. Let $x,y \in \RR^n$ be arbitrary. We see
  that for some $i$, $\rho(x,y) = |x_i-y_i|$. By $4.B$, we have that 
  $|x_i-y_i| = |y_i-x_i|$. By the definition of max, we know that $i$
  is such that $|x_i-y_i| \geq |x_j-y_j|$ for any $j$, and thus, 
 $|y_i-x_i|\geq |y_j-x_j|$, and we have that $\rho(y,x)=|y_i-x_i|$
 We thus have that $\rho(x,y)=|x_i-y_i| = |y_i-x_i|=\rho(y,x)$. As
 $x,y$were arbitrary, we see that for all $x,y$, $\rho(x,y) =
 \rho(y,x)$

 We will finally show that for any $x,y,z\in \RR^n$, that
 $\rho(x,y)\leq \rho(x,z)+ \rho(z,y)$. Let $x,y,z \in \RR^n$ be aribtrary.We see that for some $i$, 
 $\rho(x,y) = |x_i-y_i|$, and by $4.B$, we know that $ |x_i-y_i| \leq
 |x_i-z_i| +|z_i-y_i|$, and by definition of max, we have that
 $\rho(x,z)\geq |x_i-z_i|$, and  $\rho(z,y)\geq |z_i-y_i|$. We thus
 see that $\rho(x,y)\leq \rho(x,z)+ \rho(z,y)$, and as $x,y,z$ were
 arbitrary, we have that this is true for any $x,y,z\in \RR^n$

 As we have show that all properties of a metric hold, we know that 
 $\RR^n \times \RR^n : \rightarrow \RR_+: (x,y) \mapsto
  max_{i=1,...,n} |x_i-y_i|$ is a metric.
\end{proof}

Primary Author: Reilly Noonan Grant

\begin{minorEx}%4.2
  Prove that $\RR^n \times \RR^n :\rightarrow \RR_+: (x,y) \mapsto
  \Sigma_{i=1,...,n} |x_i- y_i|$ is a metric.
\end{minorEx}

\begin{proof}
  We will first show that if $\rho(x,y)=0$, that $x=y$. Let $x,y\in
  \RR^n$ be arbitrary. We see that
  for any $i$, $|x_i-y_i|\geq 0$ by the definition of absolute value,
  and thus as$\rho(x,y) =0$ we have that for every $i$,
  $|x_i-y_i|=0$, and thus by $4.B$, $x_i=y_i$, and thus $x=y$. As
  $x,y$ were arbitrary, we see that this is true for any $x,y \in
  \RR^n$. We can also see that if $x=y$, that for each $i$ $x_i=y_i$,
  and thus $|x_i-y_i| = 0$, and thus $\rho(x,y)=0$. We see as both
  directions hold that $\rho(x,y)=0$ if and only if $x=y$.

  We will now show that for any $x,y$, $\rho(x,y)=\rho(y,x)$. Let
  $x,y\in \RR^n$ be arbitrary. We see that $\rho(x,y)= \Sigma_{i=1}^n
  |x_i-y_i| = \Sigma_{i=1}^n |y_i - x_i|= \rho(y,x)$, and as $x,y \in
  \RR^n$ were arbitrary, we have that $\rho(x,y)=\rho(y,x)$ for all
  $x,y\in \RR^n$

  Finally, we will show that for any $x,y,z \in \RR^n$ that 
  $\rho(x,y) \leq \rho(x,z) + \rho(z,y)$. We see that 
  $$\rho(x,y) = \Sigma_{i=1}^n |x_i-y_i|=\Sigma_{i=1}^n
  |x_i-z_i+z_i-y_i| $$
  and by the triangle inequality, 
    $$\Sigma_{i=1}^n |x_i-z_i+z_i-y_i| \leq  \Sigma_{i=1}^n
    |x_i-z_i|+|z_i-y_i|$$
    We now see by spliting up the sums, that we have that 
    $$\Sigma_{i=1}^n  |x_i-z_i|+|z_i-y_i| = \Sigma_{i=1}^n |x_i-z_i|
    +\Sigma_{i=1}^n |z_i-y_i|=\rho(x,z)+\rho(z,y)$$
    and thus $\rho(x,y) \leq \rho(x,z)+\rho(z,y)$. As $x,y,z$ were
    arbitrary, we have that this is true for any $x,y,z \in \RR^n$.

    As we have shown all properties of a metrics hold, we know that 
    $\RR^n \times \RR^n :\rightarrow \RR_+: (x,y) \mapsto
  \Sigma_{i=1,...,n} |x_i- y_i|$ is a metric.
\end{proof}

Primary Author: Reilly Noonan Grant


\begin{minorEx}%4.3
  Prove that $\rho^{p}$ is a metric for any $p \geq 1$, where

$$\rho^{(p)} : (x,y) \mapsto (\sum_{i=1}^n |x_i - y_i|^p)^{1/p} \text{,   }p\geq 1$$

and then show the Holder Inequality. 

Let $x_1,...,x_n,y_1,...,y_n \geq 0$, let $p,q >0$,
and let $1/p + 1/q =1$. Prove that
$$\sum_{i=1}^n x_iy_i \leq (\sum_{i=1}^n x_i^p)^{1/p} (\sum_{i=1}^ny_i^q)^{1/q}$$
\end{minorEx}

\begin{proof}
  We will first show that if $\rho^{(p)}(x,y)=0$ then $x=y$. Let $x,y
  \in \RR^n$ be arbitrary, such that $\rho^{(p)}(x,y)=0$. We then have that 
  $$(\sum_{i=1}^n |x_i - y_i|^p)^{1/p} =0$$
  and thus
  $$\sum_{i=1}^n |x_i - y_i|^p =0$$
  As $|x_i - y_i|^p$ is never negative for any $x_i,y_i$, we thus see
  that for every $i$
  $$|x_i - y_i|^p =0$$ and the exponent function is injective, we know
  that $|x_i - y_i|=0$, and thus $x_i=y_i$ for all $i$, and thus by
  definition of equality for $\RR^n$, we have $x=y$. As $x,y$ were
  arbitrary, we know have that  if $\rho^{(p)}(x,y)=0$ then $x=y$.

  We also see by the definition of equality on $\RR^n$, that if for
  some $x,y\in \RR^n$ $x=y$, that for
  every $i$, $x_i=y_i$, and thus $|x_i-y_i|=0$ for every $i$, and thus
  $(\sum_{i=1}^n |x_i - y_i|^p)^{1/p}=0$  and thus
  $\rho^{(p)}(x,y)=0$. We thus have that as $x,y$ were arbitrary, that
  if $x=y$, then $\rho^{(p)}(x,y)=0$, and thus $\rho^{(p)}(x,y)=0$ iff
  $x=y$.

  We will now show that for any $x,y\in \RR^n$ that
  $\rho(x,y)=\rho(y,x)$. Let $x,y\in \RR^n$ be arbitrary.
  We see that 
  $$\rho(x,y) = 
  (\sum_{i=1}^n |x_i - y_i|^p)^{1/p}= 
  (\sum_{i=1}^n |y_i - x_i|^p)^{1/p}= 
  \rho(x,y)$$
  and as $x,y$ were arbitray, we have that 
   for any $x,y\in \RR^n$ that
  $\rho(x,y)=\rho(y,x)$.

  We now will show that $\rho(x,y) \leq \rho(x,z) + \rho(z,y)$ for any
  $x,y,z$. We see by refering to ``When Cauchy and Holder Met
  Minkowski: a Tour through well-known inequalities'' that this is true.
\end{proof}
Primary Author: Reilly Noonan Grant

\begin{minorEx}%4.4
    [Riddle]
    How is this related to $\Sigma^2$, $\Sigma^{\infty}$, and $\Sigma^1$ from
    Section 3?
\end{minorEx}

\begin{proof}[Answer]
  As $\Sigma^2,\Sigma^\infty$, and $\Sigma^1$ were respectively the
  set of all possible open disks,  the set of all possible open
  squares with sides parallel to the coordinate axes,
  and the set of all possible open squares with sides
  parallel to the bisectors of the coordinate angles, and the metrics 
  $\rho^2,\rho^\infty$, and $\rho^1$ define open balls that are
 open disks, open squares parallel to the coordinate axes, and open
 squares parallel to the bisectors of the coordinate angles, that each
 element of these bases can be defined as a ball around a certain
 point with a certain radius, that each metric defines their
 corresponding basis, in $\RR^2$.
\end{proof}

Primary Author: Reilly Noonan Grant

\begin{minorEx}%4.5
    Let $p \geq 1$. Prove that for any two sequences $x, y \in l^{(p)}$ the
    series $\sum_{i=1}^{\infty} \abs{x_i - y_i}^p$ converges and that
    \[
        (x, y) \mapsto \left( \sum_{i=1}^{\infty} \abs{x_i - y_i}^p
        \right)^{1/p}
    \]
    is a metric on $l^{(p)}$.
\end{minorEx}
