\subsection{Metrizable Topological Space}

\begin{majorEx}%4.J
    An indiscrete space is not metrizable if it is not a singleton (otherwise,
    it has too few open sets).
\end{majorEx}

\begin{proof}
   Let $X$ be an arbitary nonempty indiscrete space.
   We will show that if $X$ metrizable by some metric $\rho$, that it
   is a singleton. Let $a,b \in X$ be arbitrary. We see that if
   $\rho(a,b)>0$, that $b \notin B_{\rho(a,b)}(a)$, and $a \notin
   B_{\rho(a,b)}(a)$, and as $a \in B_{\rho(a,b)}(a)$, and $b \in
   B_{\rho(a,b)}(b)$, we would have that $B_{\rho(a,b)}(a)\neq
   B_{\rho(a,b)}(b)\neq \emptyset$, and thus there would be three sets
   in the topology. We thus have that $\rho(a,b)=0$, and thus that
   $a=b$. As $a,b$ were arbitrary, we see that there is only one
   element in $X$, and thus that $X$ is a singleton.
\end{proof}

\begin{majorEx}%4.K
    A finite space $X$ is metrizable iff it is discrete.
\end{majorEx}

\begin{proof}
  Let $X$ be an arbitrary finite Topological space. Suppose $X$ is
  discrete. We would then see that the metric $4.A$ would generate it,
  by $4.19$. Now suppose that $X$ is metrizable for some metric
  $\rho$. Let $a\in X$ be 
  arbitrary such that $a$. We see that as $X$ is a finite set, that
  $B_r(a)$ where $r = inf (\rho(a,b)| b\neq a, b\in X )$ is in the
  metric topology of $X$, and will only contain $a$ by the definition
  of an open ball. As $a$ was arbitrary, we see that every individual
  element of $X$ has a set which contains only it in the topology, and
  thus we see that $X$ is discrete, as any supset of $X$ can be made
  from the union of individual elements.
\end{proof}


\begin{minorEx}%4.25
    Which of the topological spaces described in Section 2 are metrizable?
\end{minorEx}


\begin{proof}[Answers]
  To determine this, we will examine the topological spaces described
  in 2.1, 2.3, 2.C , 2.6.

  We first assume that there is a metric $\rho$ such that the
  topological space in $2.1$ is induced by it. We see that in the
  arrow topology, that if $1$ is in a set, that $2$ is also in a set,
  however, we see that as $1\neq 2$, that $\rho(1,2)>0$, and thus
  $\B_{\rho(1,2)}(1)$ contains $1$, but not $2$ we thus have a
  contradiction and see that $2.1$ is not metrizable.

  We see that as 2.3 is finite, we know by $4.K$ that it is
  metrizable, with the metric described in $4.A$.

  We see by $4.H$ that the topological space described by $2.C$ is
  metrizable.

  We now assume a metric $\rho$ such that the
  topological space in $2.6$ is induced by it. We see for some point
  $a$, $a$ is in every set. However, we see that if for some $b$,
  $\rho(a,b)>0$, that there is a set $B_{\rho(a,b)}(b)$ which does not
  contain $b$, but which is not the empty set. We also see that as
  $2.6$ is obtained by adding  a point to a topology, that there must
  be at least one point $b \neq a$, and thus we have a contradiction,
  and the topological space in $2.6$ is not metrizable.
\end{proof}