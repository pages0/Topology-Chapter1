\subsection{Ultrametrics and p-Adic Numbers}

\begin{majorEx} 4.Rx
  Check that only one metric in 4.A -4.2 is an ultra metric
\end{majorEx}

To do this, we must examine the following metrics

4.A:
$\rho: X\times X \rightarrow \RR_+: (x,y) \mapsto $
\[ \begin{cases} 
    0 & \text{if } x=y \\
    1 & \text{if } x \neq y 
  \end{cases}
  \]

4.B: $\RR \time \RR \rightarrow \RR_+ :(x,y) \mapsto |x-y|$

4.C: $\RR^n \times \RR^n \rightarrow \RR_+ : (x,y) \mapsto \sqrt{\Sigma_{i=1}^n (x_i-y_i)^2}$

4.1: $\RR^n \times \RR^n \rightarrow \RR_+ : (x,y) \mapsto max_{i=1,...,n}|x_i-y_i$ 

4.2 $\RR^n \times \RR^n \rightarrow \RR_+ : (x,y) \mapsto \Sigma_{i=1}^n |x_i - y_i|$

We will first show that $4.A$ is an ultrametric, and that all other metrics have a counter example.

\begin{proof}
  We assume for sake of contradiction that $4.A$ is not an ultra metric, and thus 
  that for some $x,y,z$ $\rho(x,y) > max \{\rho(x,z) , \rho(z,y)\}$. We thus see 
  that as there are only two possible values for $\rho$, that $\rho(x,y)=1$, and 
  $max \{\rho(x,z) , \rho(z,y)\}=0$ and thus $\rho(z,y)=0$ and $\rho(x,z)=0$. 
  However, we see that as $\rho(z,y)=0$, and $\rho(x,z)=0$ that by definition of 
  the metric, that $z=y$, and $x=z$, and thus $x=y$, and by definitiion of $\rho$, we must have that $\rho(x,y)=0$. We thus have a contradiction, and we know that $4.A$ is an an ultra metric
\end{proof}

We now see that if $x=1,y=0, z=.5$, that $|x-y|= 1$, $|x-z|=0.5$ and $|z-y|=0.5$. and thus for the metric in $4.B$, there is some value of $x,y$ and $z$ such that
$\rho(x,y) > max \{\rho(x,z) , \rho(z,y)\}$.



