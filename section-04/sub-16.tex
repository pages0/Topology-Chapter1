\subsection{Ultrametrics and $p$-Adic Numbers}

\begin{majorEx}%4.Rx
  Check that only one metric in 4.A -4.2 is an ultra metric
\end{majorEx}

To do this, we must examine the following metrics

4.A:
$\rho: X\times X \rightarrow \RR_+: (x,y) \mapsto $
\[ \begin{cases} 
    0 & \text{if } x=y \\
    1 & \text{if } x \neq y 
  \end{cases}
  \]

4.B: $\RR \times \RR \rightarrow \RR_+ :(x,y) \mapsto |x-y|$

4.C: $\RR^n \times \RR^n \rightarrow \RR_+ : (x,y) \mapsto \sqrt{\Sigma_{i=1}^n (x_i-y_i)^2}$

4.1: $\RR^n \times \RR^n \rightarrow \RR_+ : (x,y) \mapsto \max_{i=1,...,n}|x_i-y_i$ 

4.2 $\RR^n \times \RR^n \rightarrow \RR_+ : (x,y) \mapsto \Sigma_{i=1}^n |x_i - y_i|$

We will first show that $4.A$ is an ultrametric, and that all other metrics have a counter example.

\begin{proof}
  We assume for sake of contradiction that $4.A$ is not an ultra metric, and thus 
  that for some $x,y,z$ $\rho(x,y) > \max \{\rho(x,z) , \rho(z,y)\}$. We thus see 
  that as there are only two possible values for $\rho$, that $\rho(x,y)=1$, and 
  $max \{\rho(x,z) , \rho(z,y)\}=0$ and thus $\rho(z,y)=0$ and $\rho(x,z)=0$. 
  However, we see that as $\rho(z,y)=0$, and $\rho(x,z)=0$ that by definition of 
  the metric, that $z=y$, and $x=z$, and thus $x=y$, and by definitiion of $\rho$, we must have that $\rho(x,y)=0$. We thus have a contradiction, and we know that $4.A$ is an an ultra metric
\end{proof}

We now see that if $x=1,y=0, z=.5$, that $|x-y|= 1$, $|x-z|=0.5$ and $|z-y|=0.5$, and thus for the metric in $4.B$, there is some value of $x,y$ and $z$ such that
$\rho(x,y) > max \{\rho(x,z) , \rho(z,y)\}$.

We also see that given the same values of $x,y$, and $z$ that
$\sqrt{(x-y)^2}=1$, $\sqrt{(x-z)^2}= 0.5$ $\sqrt{(y-z)^2}= 0.5$, and
thus for the metric in $4.C$, there is some value of $x,y$ and $z$ such that
$\rho(x,y) > max \{\rho(x,z) , \rho(z,y)\}$.

We also see that in the case of $n=1$, that 4.1 and 4.2 reduce to the
cases above, and thus are not ultra metrics.

\begin{majorEx}%4.Sx
    Prove that all triangles in an ultrametric space are isosceles (i.e., for
    any three points $a,b,$ and $c$, at least two of the three distances
    $\rho(a,b)$, $\rho(b,c)$, and $\rho(a, c)$ are equal).
\end{majorEx}

\begin{proof}
    Suppose for the sake of contradiction that $\rho$ is an ultrametric and $a
    b, c$ do not form an isosceles triangle. Then without loss of
    generality, we assume that
    \[
        \rho(a,b) < \rho(a,c) < \rho(b,c).
    \]
    We would then have that 
    $\rho(b,c)>\rho(a,c) $, and $\rho(b,c)>\rho(a,b)$, and thus
    $\rho(b,c)> max\{\rho(a,c), \rho(a,b)\}$. We thus have that
    this contradicts that $\rho$ is an ultrametrics, and thus we have
    that there is not a strict inequality on $\rho(a,b)$,$\rho(a,c)$,
    and $\rho(b,c)$, and thus at least two of the three distances 
    are equal.    

\end{proof}

\begin{majorEx}%4.Tx
    Prove that spheres in an ultrametric space are not only closed (see Problem
    4.23), but also open.
\end{majorEx}

\begin{proof}
    Let $a \in X$, $r \in \RR_+$ be arbitrary and consider the sphere $S_r(a)$.
    It suffices to show that for every $x \in S_r(a)$, there is a ball $B_{r'}(x)
    \subseteq S_r(a)$. 

    Let $x \in S_r(a)$ be arbitrary, and choose $r' = \frac{r}{2}$. We
    also let $y\in B_{r'}(x)$be arbitrary. We see that we have
    \begin{align*}
        \rho(a, y) &\leq \max \set{\rho(a, x), \rho(x, y} \\
        &= \max \set{r, \frac{r}{2}} \\
        &= r,
    \end{align*}
    and we have
    \begin{align*}
        \rho(a, x) &\leq \max \set{\rho(a, y), \rho(x, y)} \\
        r &\leq \max \set{\rho(a, y), \frac{r}{2}} \\
        r &\leq \rho(a, y).
    \end{align*}
    Hence,
    \[
        r \leq \rho(a, y) \leq r,
    \]
    so $r = \rho(a, y)$. Thus, $y \in S_{r}(a)$, and as $y$ was
    arbitrary, we have that $B_{r'}(x) \subseteq S_r(a)$. As $x$ was
    arbitrary, we see that this is true for every $x \in S_r(a)$, and
    thus as the union of open balls is an open set, we have as that every
    element of $S_r(a)$ is contained in an open set that is a subset
    of $S_r(a)$, that $S_r(a)$ is an open set.

\end{proof}


\begin{majorEx}
    Prove that $\rho$ is an ultrametric.
\end{majorEx}

\begin{proof}
    Let $x, y, z \in \QQ$ be arbitrary. There exist $r_i, s_i, \alpha_i \in \ZZ$
    ($i = 1, 2, 3$) such that
    \begin{align*}
        x - y &= \frac{r_1}{s_1} p^{\alpha_1} \\
        x - z &= \frac{r_2}{s_2} p^{\alpha_2} \\
        y - z &= \frac{r_3}{s_3} p^{\alpha_3}.
    \end{align*}
    Note that each $\alpha_i$ is unique to the particular difference.
    Without loss of generality, assume $\alpha_2 \geq \alpha_1$. If $\alpha_1 =
    \alpha_3$, we are done, because $\rho(x,z) \leq \max\set{\rho(x,y), \rho(y,z)}$,
    the other inequalities following easily. We have
    \begin{align*}
        y - z &= (y - x) + (x - z) \\
        \frac{r_3}{s_3}p^{\alpha_3} &= \frac{r_2}{s_2} p^{\alpha_2} -
        \frac{r_1}{s_1} p^{\alpha_1} \\
        \frac{r_3}{s_3}p^{\alpha_3} &= \left(\frac{r_2}{s_2} p^{\alpha_2 -
        \alpha_1} - \frac{r_1}{s_1} \right) p^{\alpha_1} \\
        \frac{r_3}{s_3}p^{\alpha_3} &= \left(\frac{r_2p^{\alpha_2 -
        \alpha_1} - r_1}{s_1s_2} \right) p^{\alpha_1}.
     \end{align*}
     It now suffices to show that 
     \[
         \gcd \left(\frac{r_2p^{\alpha_2 - \alpha_1} - r_1}{s_1s_2},
         p \right) = 1.
     \]
     To this end, notice that the denominator is coprime with $p$, since both
     $s_1$ and $s_2$ are coprime with $p$. And notice that the numerator is
     coprime with $p$, since $r_1$ is coprime with $p$. Hence, we can write
     \[
         \frac{r_3}{s_3}p^{\alpha_3} = \frac{r_4}{s_4} p^{\alpha_1}
     \]
     where $r_4 = r_2 p^{\alpha_2 - \alpha_1} - r_1$ and $s_4 = s_1 s_2$ are
     coprime with $p$. This implies that $\alpha_1 = \alpha_3$, and the result
     follows.
\end{proof}
