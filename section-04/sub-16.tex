\subsection{Ultrametrics and $p$-Adic Numbers}

\begin{majorEx}
    Check that only one metric in $4.A-4.2$ is an ultrametric. Which one?
\end{majorEx}

\begin{proof}[Answer]
    The metric $4.A$ given by
    \[
        \rho(x,y) = 
        \begin{cases}
            0 & x = y \\
            1 & x \ne y,
        \end{cases}
    \]
    is an ultrametric. The rest are not. For example, in $\RR^2$ the points $x =
    (0,0)$, $y = (3,0)$ and $z = (0, 4)$ have
    \[
        \rho(y,z) = 5 \geq \max\set{4, 3} = \max{\rho(x,z), \rho(x,y)}.
    \]
\end{proof}

\begin{majorEx}
    Prove that all triangles in an ultrametric space are isosceles (i.e., for
    any three points $a,b,$ and $c$, at least two of the three distances
    $\rho(a,b)$, $\rho(b,c)$, and $\rho(a, c)$ are equal).
\end{majorEx}

\begin{proof}
    Suppose for the sake of contradiction that $\rho$ is an ultrametric and $a,
    b, c$ do not form an isosceles triangle. Then without loss of generality,
    assume that
    \[
        \rho(a,b) < \rho(a,c) < \rho(b,c).
    \]
    Then
    \[
        \rho(b,c) > \max\set{\rho(a,b), \rho(b,c)},
    \]
    contradicting the claim that $\rho$ is an ultrametric.
\end{proof}

\begin{majorEx}
    Prove that spheres in an ultrametric space are not only closed (see Problem
    4.23), but also open.
\end{majorEx}

\begin{proof}
    Let $a \in X$, $r \in \RR_+$ be arbitrary and consider the sphere $S_r(a)$.
    It suffices to show that for every $x \in S_r(a)$, there is a ball $B_{r'}
    \subseteq S_r(a)$. 

    Let $x \in S_r(a)$ be arbitrary, and choose $r' = \frac{r}{2}$. For any $y
    \in B_{r'}(x)$, we have
    \begin{align*}
        \rho(a, y) &\leq \max \set{\rho(a, x), \rho(x, y} \\
        &= \max \set{r, \frac{r}{2}} \\
        &= r,
    \end{align*}
    and we have
    \begin{align*}
        \rho(a, x) &\leq \max \set{\rho(a, y), \rho(x, y)} \\
        r &\leq \max \set{\rho(a, y), \frac{r}{2}} \\
        r &\leq \rho(a, y).
    \end{align*}
    Hence,
    \[
        r \leq \rho(a, y) \leq r,
    \]
    so $r = \rho(a, y)$. Thus, $y \in S_{r}(a)$, and the result follows.
\end{proof}


\begin{majorEx}
    Prove that $\rho$ is an ultrametric.
\end{majorEx}

\begin{proof}
    Let $x, y, z \in \QQ$ be arbitrary. There exist $r_i, s_i, \alpha_i \in \ZZ$
    ($i = 1, 2, 3$) such that
    \begin{align*}
        x - y &= \frac{r_1}{s_1} p^{\alpha_1} \\
        x - z &= \frac{r_2}{s_2} p^{\alpha_2} \\
        y - z &= \frac{r_3}{s_3} p^{\alpha_3}.
    \end{align*}
    Note that each $\alpha_i$ is unique to the particular difference.
    Without loss of generality, assume $\alpha_2 \geq \alpha_1$. If $\alpha_1 =
    \alpha_3$, we are done, because $\rho(x,z) \leq \max\set{\rho(x,y), \rho(y,z)}$,
    the other inequalities following easily. We have
    \begin{align*}
        y - z &= (y - x) + (x - z) \\
        \frac{r_3}{s_3}p^{\alpha_3} &= \frac{r_2}{s_2} p^{\alpha_2} -
        \frac{r_1}{s_1} p^{\alpha_1} \\
        \frac{r_3}{s_3}p^{\alpha_3} &= \left(\frac{r_2}{s_2} p^{\alpha_2 -
        \alpha_1} - \frac{r_1}{s_1} \right) p^{\alpha_1} \\
        \frac{r_3}{s_3}p^{\alpha_3} &= \left(\frac{r_2p^{\alpha_2 -
        \alpha_1} - r_1}{s_1s_2} \right) p^{\alpha_1}.
     \end{align*}
     It now suffices to show that 
     \[
         \gcd \left(\frac{r_2p^{\alpha_2 - \alpha_1} - r_1}{s_1s_2},
         p \right) = 1.
     \]
     To this end, notice that the denominator is coprime with $p$, since both
     $s_1$ and $s_2$ are coprime with $p$. And notice that the numerator is
     coprime with $p$, since $r_1$ is coprime with $p$. Hence, we can write
     \[
         \frac{r_3}{s_3}p^{\alpha_3} = \frac{r_4}{s_4} p^{\alpha_1}
     \]
     where $r_4 = r_2 p^{\alpha_2 - \alpha_1} - r_1$ and $s_4 = s_1 s_2$ are
     coprime with $p$. This implies that $\alpha_1 = \alpha_3$, and the result
     follows.
\end{proof}
