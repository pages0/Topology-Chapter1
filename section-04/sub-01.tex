\subsection{Definition and First Examples}

\begin{majorEx}%4.A
  Prove that the function
  \begin{center}
  $\rho : X \times X \rightarrow \RR_+ : (x,y) \mapsto $
  \[ \begin{cases} 
    0 & \text{if } x=y \\
    1 & \text{if } x \neq y 
  \end{cases}
  \]
\end{center}
\end{majorEx}
is a metric for any set $X$
\begin{proof}
   We first see that as if $x=y$, then $\rho(x,y)$, and if $x\neq y$, that $\rho(x,y)=1\neq 0$, that we have that the first property of metrics holds.

   We now see that as $x=y$ implies $y=x$, that if $x=y$, 
   $\rho(x,y)=\rho(y,x)=0$, and if $x\neq y$, $\rho(x,y)=\rho(y,x)=1$, and thus the 
   second property of metrics holds.

   Finally, we have that as equality is a equivalence relation, that if $x=y$ , and $y=z$, 
   that $x=z$, and thus $\rho(x,y) = \rho(x,z)+ \rho(z,y)=0$, and that otherwise, as $\rho(x,y)\geq 0$ for any $x,y$, we see that if $x\neq y$, that $\rho(x,y) \leq \rho(x,z)+ \rho(z,y)$   
\end{proof}  
  

\begin{majorEx} %4.B
  Prove that $\RR \times \RR \rightarrow  \RR_+ : (x,y) \mapsto |x-y|$ is a metric
\end{majorEx}

\begin{proof}
  We first can see that for any $x,y \in \RR, |x-y|=0$ is only true when $x-y=0$, and thus
  when $x=y$. We thus have that $\rho(x,y)=0$ is true if and only if $x=y$
  
  We also can see that as for any $x,y\in \RR$, we have that if
  $x-y=a$, for some $a\in \RR$ that  $y-x={-a}$. By the definition of
  absolute value, we know that $|a| = |{-a}|$, and thus for any $x,y$,
  we have that $\rho(x,y) = \rho(y,x)$. As $x,y$ were arbitrary, we
  have that for all $x,y$ $\rho(x,y) = \rho(y,x)$.

  We finally will prove that $\rho(x,y) \leq \rho(x,z) + \rho(z,y)$.
  Let $x,y,z \in \RR$ be arbitrary. We see that $\rho(x,y)= |x-y| = |x-z+z-y|$,
  and by the triangle inequality, we know that $|x-z+z-y|\leq |x-z| +
  |z-y|=\rho(x,z)+ \rho(z,y)$. We thus have that $\rho(x,y) \leq
  \rho(x,z) + \rho(z,y)$, and as $x,y,z$ were arbitrary, that this is
  true for any $x,y,z$.
\end{proof}
