\subsection{Norms and Normed Spaces}

\begin{minorEx}%4.15
    Prove that if $x \to \norm{x}$ is a norm, then
    \[
        \rho : X \times X \to \RR_+ : (x,y) \mapsto \norm{x - y}
    \]
\end{minorEx}

\begin{proof}
  We will first show that $\rho(x,y) =0$ if and only if $x=y$.Let
  $x,y\in X$ be arbitrary.
  We see that as $\rho(x,y)= \norm{x-y}$, that $\rho(x,y)=0$ 
  if and only if $x-y=0$, and thus if $x=y$. We thus have
  that $\rho(x,y) =0$ if and only if $x=y$, and as $x,y$ were
  arbitrary, this is true for all $x,y\in X$.

  We will now show that $\rho(x,y)= \rho(y,x)$. Let $x,y\in X$ be
  arbitrary. We see that $\rho(x,y)
  = \norm{x-y}=\norm{({-1})(y-x)}=\norm{y-x}= \rho(y,x)$. As $x,y$
  were arbitrary, we see that $\rho(x,y)= \rho(y,x)$ for all $x,y$.

  We will now show that $\rho(x,y) \leq \rho(x,z) + \rho(z,y)$ for all
  $x,y,z \in X$. Let $x,y,z$ be arbitrary. We see that 
  $\rho(x,y) = \norm{x-y} =
  \norm{x-z +z -y} \leq \norm{x-z}+\norm{z-y}= \rho(x,z) + \rho(z,y)$.
  As $x,y,z\in X$ were arbitrary, we see that for any $x,y,z\in X$,
  that $\rho(x,y) \leq \rho(x,z) + \rho(z,y)$
\end{proof}

Primary Author: Reilly Noonan Grant

\begin{minorEx}%4.16
    Look through the problems of this section and figure out which 
    of the metric spaces involved are, in fact, normed vector spaces.
\end{minorEx}


\begin{proof}[Answer]
  We will look through the metric spaces defined by $4.A$, $4.B$,
  $4.C$, $4.1$, $4.2$ and $4.3$ to determine which of the metric
  spaces defined by these 
  metrics are normed vector spaces. We will do this, by determining
  for each metric $\rho$, if $\rho(x,y)= \norm{x-y}$ for some norm.
  Before examining any specific metric, we
  see that for any metric space defined as in $4.15$, that 
  $\rho(x,0)=\norm{x}$, and thus to examine the
  properties $1$ and $2$ of a norm, we can examine $\rho(x,0)$ for
  every $x\in X$. We see that as $\rho(x,0)=0$ if and only if $x=0$,
  that property $1$ will hold for every metric. We also see that if
  property $2$ of a norm holds, that
  $\norm{x+y}=\rho(x,-y) \leq \norm{x}+\norm{-y}=\norm{x} + \norm{y}$
  and thus property $3$ of a norm holds. We thus see that we must only
  examine property $2$ of a norm to determine if  a metric is defined
  by a norm as in $4.15$.

  We first examine $\rho$ defined by  $4.A$. We see that $2\cdot\rho(1,0)=1$, and
  $\rho(2,0)=1$, and thus property $2$ of norms does not hold.

  We now examine  $\rho$ defined by  $4.B$. We see that $|\lambda x|= |\lambda||x|$ by
  definition of absolute value, and thus that the metric space defined
  by the metric in $4.B$ is a normed vector space, as absolute value
  is a norm. 

  We now examine $\rho$ defined by $4.C$. We let $x \in \RR^n$ be arbitrary for an
  arbitrary $n$. We see that for an arbitrary $\lambda \in \RR$
  $\rho(\lambda x,0) = \sqrt{\Sigma_{i=1}^n (\lambda x_i)^2 }=
  \sqrt{\Sigma_{i=1}^n \lambda^2 x_i^2 }=
  \sqrt{\lambda^2 \Sigma_{i=1}^n x_i^2 } =
  \lambda \sqrt{\Sigma_{i=1}^n x_i^2 }= |\lambda|\rho(x,0)$. As $x$,
  $n$ and $\lambda$ were arbitrary, we have that 
  property $2$ of a norm holds, and as $\rho$ is a metric, we know that
  the metric space defined by $\rho$ is a normed vector space.

  We now examine $\rho$ defined by $4.1$. 
  We let $x\in \RR^n$ be arbitrary for an arbitrary $n$. We see that 
  for an arbitrary $\lambda \in \RR$, $\rho(\lambda x,0)=
  max_{i=1,...,n} |\lambda x_i| =
  max_{i=1,...,n} |\lambda| | x_i|=
  |\lambda| max_{i=1,...,n} | x_i|=
  |\lambda| \rho(x,0)$. As $x$  $n$ and $\lambda$ were arbitrary,
  we have that property $2$ of a norm holds, and as $\rho$ is a
  metric, we know that the metric space defined by $\rho$ is a normed
  vector space. 

  We now examine $\rho$ defined by $4.2$. 
  We let $x\in \RR^n$ be arbitrary for an arbitrary $n$. We see that 
  for an arbitrary $\lambda \in \RR$, $\rho(\lambda x,0)=
  \Sigma_{i=1}^n|\lambda x_i| =
  \Sigma_{i=1}^n |\lambda| | x_i|=
  |\lambda| \Sigma_{i=1}^n | x_i|=
  |\lambda| \rho(x,0)$. As $x$  $n$ and $\lambda$ were arbitrary,
  we have that property $2$ of a norm holds, and as $\rho$ is a
  metric, we know that the metric space defined by $\rho$ is a normed
  vector space. 

  We now examine $\rho$ defined by $4.3$. 
  We let $p\in \NN$ and $x\in \RR^n$ be arbitrary
  for an arbitrary $n$. We see that 
  for an arbitrary $\lambda \in \RR$, $\rho(\lambda x,0)=
  (\Sigma_{i=1}^n |\lambda x_i|^p)^{1/p} =
  (\Sigma_{i=1}^n |\lambda|^p |x_i|^p)^{1/p} =
 ( |\lambda|^p|\Sigma_{i=1}^n |x_i|^p)^{1/p} =
 ( |\lambda|^p)^{1/p}(\Sigma_{i=1}^n |\lambda x_i|^p)^{1/p} =
  |\lambda| \rho(x,0)$. As $x$,$p$, $n$ and $\lambda$ were arbitrary,
  we have that property $2$ of a norm holds, and as $\rho$ is a
  metric, we know that the metric space defined by $\rho$ is a normed
  vector space. 
\end{proof}
Primary Author: Reilly Noonan Grant

\begin{minorEx}%4.17
    Prove that every ball in a normed space is a convex set symmetric with
    respect to the center of the ball.
\end{minorEx}

\begin{proof}
  We first let $B_r(a)$ be an arbitary ball in a normed space. We will
  first show that it is a convex set, and thus if $x,y\in B_r(a)$,
  that for any $z$ which is in the segment bettween $x$ and $y$
  that $z\in B_r(a)$. Let $x,y\in B_r(a)$ be arbitrary, and let $z$
  such that $\rho(x,z)+\rho(z,y)=\rho(x,y)$ be arbitrary. We see that
  $z= \lambda x + (1-\lamba)y$ for some $\lambda \in (0,1)$, by
  definition of a segment in a vector space. We see that 
  $\norm{z-a} = \norm{\lambda x + (1-\lambda)y - a} = \norm{\lambda
    x+ (1-\lambda)y + \lambda a + (1-\lambda) a} \leq
  \lambda\norm{x-a} + (1-\lambda)\norm{y-a}< \lambda r + (1-\lambda)r
  = r$. We thus have that $\rho(a,z)< r$, and thus that $z\in B_r(a)$,
  and thus that $B_r(a)$ is convex. As $B_r(a)$ was arbitrary, we know
  that every ball in a normed space is a convex set.

  We will now show that every ball in a normed space is 
  symmetric with respect to the center of the ball.  Let $B_r(a)$ be
  an arbitary ball in a normed space. We will show that for any $x \in
  B_r(a)$, that the vector reflected around $a$, $2a -x$ is in
  $B_r(a)$. 
  We see that $\rho(2a-x,a)= \norm{2a-x -a} = \norm{a-x}=\rho(x,a)<r$,
  and thus $2a -x\in B_r(a)$. As $B_r(a)$ was arbitrary, we know that 
   every ball in a normed space is symmetric with respect to the
   center of the ball. 

   As we have proven all properties, we see that every ball in a
   normed space is a convex set symmetric with respect to the center
   of the ball.
\end{proof}
Primary Author: Reilly Noonan Grant

\begin{minorEx}
    Prove that every convex closed bounded set in $\RR^n$ that has a center of
    symmetry and is not contained in any affine space except $\RR^n$ itself is a
    unit ball with respect to a certain norm, which is uniquely determined by
    this ball.
\end{minorEx}

\begin{proof}
    Let $S$ be a convex closed bounded set in $\RR^n$ with center $c$. We define
    the set operations
    \[
        \alpha \cdot S = \alpha S = \set{\alpha x : x \in S},
    \]
    and
    \[
        S + y = \set{x + y : x \in S}.
    \]
    We see that $x \in S$ if and only if $\alpha x \in \alpha S$, and $x \in S$
    if and only if $x + y \in S + y$. Let $\norm{\cdot} : \RR^n
    \to \RR_{+}$ be defined by
    \[
        \norm{y} = \inf \set{\lambda : y \in \lambda (S - c)}.
    \]
    It follows that $\norm{y} = 0$ exactly when $y = 0$, since for every
    $\lambda > 0$, $c \in \lambda S$. To see that
    \[
        \norm{\alpha y} = \abs{\alpha} \norm{y},
    \]
    it suffices to show that 
    \[
        A = \set{\lambda : \alpha x \in \lambda (S - c)} =
        \abs{\alpha} \set{\lambda : x \in \lambda (S - c)} = B.
    \]
    It would then follow that the infimum of one set is the infimum of the
    other, arriving at our claim. But $\lambda \in A$ if and only if $\alpha x \in
    \lambda(S - c)$, which holds if and only if $\lambda \in \alpha \set{\lambda
        : x \in \lambda (S - c)} = b$, as needed.

    Finally, to see the triangle inequality holds, it suffices to show that 
    \[
        \frac{x + y}{\norm{x} + \norm{y}} \in S - c,
    \]
    since this establishes that
    \[
        \frac{\norm{x + y}}{\norm{x} + \norm{y}} \leq 1,
    \]
    proving the claim. Since $S-c$ is convex, we have
    \[
        \frac{\norm{x}}{\norm{x} + \norm{y}} (S - c) + 
        \frac{\norm{y}}{\norm{x} + \norm{y}} (S - c) = 
        S - c.
    \]
    Since $\norm{x} \leq \norm{x} + \norm{y}$, we have $\frac{x}{\norm{x} +
    \norm{y}} \in S - c$. Similarly, we have $\frac{y}{\norm{x} + \norm{y}} \in
    S - c$. By the convexity of $S - c$, we conclude that 
    \[
        \frac{x + y}{\norm{x} + \norm{y}} \in S - c,
    \]
    which allows us to deduce that
    \[
        \frac{\norm{x + y}}{\norm{x} + \norm{y}} \leq 1,
    \]
    giving the desired result.
\end{proof}
