\subsection{Segments (What Is Between)}

\begin{minorEx}%4.12
    Prove that the segment with endpoints $a,b \in \RR^n$ can be described as 
    \[
        \set{x \in \RR^n : \rho(a,x) + \rho(x,b) = \rho(a, b)},
    \]
    where $\rho$ is the Euclidean metric.
\end{minorEx}

\begin{proof}
  Let $a,b \in \RR^n$ be arbitrary, and let them define a segment. Let
  $c$ be an arbitrary point on that segment. We see that the distance
  between $c$ and $a$ is given by $\rho(a,c)$, and the distance
  between $c$ and $b$ is given by $\rho(c,b)$. We also see that as $c$
  is on the segment, that the shortest distance between $a$ and $b$,
  given by $\rho(a,b)$, can also be given by moving from $a$ to $c$,
  and then from $c$ to $b$. We see that the distance given from moving
  from $a$ to $c$, and then $c$ to $b$ is also given by
  $\rho(a,c)+\rho(c,b)$, and thus $\rho(a,c)+\rho(c,b)= \rho(a,b)$. We
  thus have that if $c$ is on the segment, that $\rho(a,c)+\rho(c,b)=
  \rho(a,b)$. We can also see that if we know that $\rho(a,c)+\rho(c,b)=
  \rho(a,b)$, that we know that going from $a$ to $c$, and then $c$ to
  $b$ is the shortest distance between $a$ to $b$, and thus $c$ is on
  the segment between $a$ and $b$. As $c$ was arbitrary, and we showed
  both that $\rho(a,c)+\rho(c,b)= \rho(a,b)$ implies $c$ is on the
  segment, and that $c$ being on the segment implies that
  $\rho(a,c)+\rho(c,b)= \rho(a,b)$, we know the 
  segment with endpoints $a,b \in \RR^n$ can be described as 
    \[
        \set{x \in \RR^n : \rho(a,x) + \rho(x,b) = \rho(a, b)},
    \]
    where $\rho$ is the Euclidean metric.
\end{proof}

\begin{minorEx}%4.13
    How does the set defined as in Problem 4.12 look if $\rho$ is the metric
    defined in Problems 4.1 or 4.2? (Consider the case where $n = 2$ if it seems
    to be easier.)
\end{minorEx}

\begin{proof}[Answer]

\end{proof}