\subsection{Equivalent Metrics}

\begin{minorEx}
    Are the metrics of 4.C, 4.1, and 4.2 equivalent?
\end{minorEx}

\begin{proof}[Answer]
    These metrics are equivalent. For, any ball in 4.C can be generated by
    unions of balls in 4.1 or 4.2, and so on. This can be seen ...    
\end{proof}

\begin{minorEx}
    Prove that two metrics $\rho_1$ and $\rho_2$ in $X$ are equivalent if there
    are numbers $c, C > 0$ such that
    \[
        c \rho_1(x,y) \leq \rho_2(x,y) \leq C \rho_1(x,y)
    \]
    for any $x, y \in X$.
\end{minorEx}

\begin{proof}
    Suppose there are $c_1, c_2 > 0$ such that 
    \[
        c_1 \rho_1(x,y) \leq \rho_2(x, y) \leq c_2 \rho_1(x,y)
    \]
    for all $x, y \in X$. We show that $\rho_1$ and $\rho_2$ are equivalent. 

    Let $U \in \Omega_1$ be arbitrary. There exists some subset $\Sigma'
    \subseteq \Sigma_1$ of the base such that
    \[
        U = \bigcup_{B \in \Sigma_1} B.
    \]
    Let $B \in \Sigma_1$ be an arbitrary open ball in $\Sigma_1$. Then there are
    $r > 0$ and $x \in X$ such that $B = B^1_{r}(x)$. Let $y \in B^1_{r}(x)$ be
    arbitrary. Then
    \[
        \rho_1(x,y) < r.
    \]
    Let $r' = c_2 r$. If $z \in B_{r'}^2(y)$, then 
    \[
        \rho_2(z,y) < c_2 r.
    \]
    But $\rho_2(z,y) \leq c_2 \rho_1(z,y)$, and since
\end{proof}<++>

\begin{minorEx}
    Generally speaking, the converse is not true.
\end{minorEx}

\begin{proof}
    As we shall see in 4.33, if $\rho$ is a metric, then so is $\frac{\rho}{1 +
    \rho}$. Moreover, 4.34 shows that these are equivalent metrics. However,
    suppose that there are $c_1, c_2$ such that 
    \[
        c_1 \rho(x, y) \leq
        \frac{\rho(x, y)}{1 + \rho(x, y)} \leq
        c_2 \rho(x, y)
    \]
    for all $x, y \in X$. This is identical to the statement
    \[
        (c_1 - 1) \rho(x, y) + c_1 \rho^2(x,y) \leq 0 \leq
        (c_2 - 1) \rho(x, y) + c_2 \rho^2(x,y),
    \]
    or
    \[
        \rho(x, y) \leq 1 - \frac{1}{c_1},
        \qquad \text{and} \qquad
        \rho(x, y) \geq 1 - \frac{1}{c_2}.
    \]
    Suppose $c_1 > 0$. If $\rho(x,y) = \abs{x - y}$, for example, then
    \[
        \rho(0, 2 - \tfrac{2}{c_1}) > 1 - \frac{1}{c_1},
    \]
    so $c_1$ necessarily must not be positive. Hence, the condition does not
    hold for this combination.
\end{proof}

\begin{minorEx}
    [Riddle]
    Hence the condition of equivalence of metrics formulated in Problem 4.27 can
    be weakened. How?
\end{minorEx}

\begin{minorEx}
    The metrics $\rho^{(p)}$ in $\RR^n$ defined right before Problem 4.3 are
    equivalent.
\end{minorEx}

\begin{minorEx}
    Prove that the following two metrics $\rho_1$ and $\rho_c$ in the set of all
    continuous functions $[0,1] \to \RR$ are not equivalent:
    \[
        \rho_1(f,g) = \int_{0}^{1} \abs{f(x) - g(x)} \ dx, \qquad
        \rho_c(f,g) = \max_{x \in [0,1]} \abs{f(x) - g(x)}.
    \]
    Is it true that one of the topological structures generated by them is finer
    than the other one?
\end{minorEx}

\begin{proof}
    For $r > 0$, let $g_r \in C[0,1]$ be defined by
    \[
        g_r(x) =
        \begin{cases}
            0 & x \in [0, \tfrac{1}{2} - \tfrac{1}{2r}] \\
            2r^2(x - (\tfrac{1}{2} - \tfrac{1}{2r})) & x \in [\tfrac{1}{2} - \tfrac{1}{2r}, \tfrac{1}{2}] \\
            r - 2r^2(x - \tfrac{1}{2}) & x \in [\tfrac{1}{2},\tfrac{1}{2} + \tfrac{1}{2r}] \\
            0 & x \in [\tfrac{1}{2} + \tfrac{1}{2r}, 1]
        \end{cases},
    \]
    so that $g_r(x)$ is a ``triangular impulse'' centered at $\frac{1}{2}$ such
    that $g(\frac{1}{2}) = r$ and that the total area of the triangle is
    $\frac{1}{2}$.
    As
    \[
        \int_{0}^{1} \abs{g_r(x)} \ dx = \frac{1}{2}
    \]
    for all $r > 0$, we have that $g_r \in B_1^1(0)$, where 0 denotes the 0
    function. However, for any $f \in C[0,1]$ and $r' > 0$, by choosing $r$ such
    that
    \[
        r > \max\set{\max_{x \in [0,1]} \abs{f(x) - g(x)}, r'},
    \]
    we guarantee that $g_r \notin B_{r'}^2(f)$. Hence, $B_r^1(0)$ cannot be
    described as a union of balls with respect to $\rho_2$, so $B_r^1 \notin
    \Omega_2$.
\end{proof}
