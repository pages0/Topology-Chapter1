\subsection{Openness and Closedness of Balls and Spheres}

\begin{minorEx}
    Prove that a closed ball is closed (here and below, we mean the metric
    topology).
\end{minorEx}

\begin{proof}
    Let $(X, \Omega)$ be a space under a metric topology generated by $\rho$.
    Let $r > 0$ be arbitrary.
    We show that $D_{r}(x)$ is closed by showing that $X \setminus D_{r}(x)$ is
    open. 

    For any $y \in X \setminus D_{r}(x)$, if there is some $\varepsilon > 0$
    such that $B_{\varepsilon}(y) \subseteq X \setminus D_{r}(x)$, we can apply
    the result from 4.I to conclude that $X \setminus D_{r}(x)$ is open. To this
    end, note that as
    \[
        \rho(x, y) > r,
    \]
    since $y \notin D_{r}(x)$, there exists $\varepsilon > 0$ such that
    \[
        \rho(x, y) = r + \varepsilon.
    \]
    We show that $B_{\varepsilon}(y) \subseteq X \setminus D_{r}(x)$. Let $z \in
    B_{\varepsilon}(y)$. We have by the triangle inequality that
    \[
        \rho(x, y) \leq \rho(x, z) + \rho(z, y).
    \]
    But since $\rho(x, y) = r + \varepsilon$, and since $\rho(z, y) <
    \varepsilon$, we have that
    \[
        r + \varepsilon = \rho(x, y) \leq \rho(x, z) + \rho(z, y) < \rho(x, z) +
        \varepsilon,
    \]
    which implies that 
    \[
        r < \rho(x, z).
    \]
    Hence, $z \in X \setminus D_{r}(x)$, and the desired result follows.
\end{proof}

Primary author: David Kraemer

\begin{minorEx}
    Find a closed ball that is open.
\end{minorEx}

\begin{proof}[Answer]
    For this and the rest of the exercises in this section, we define $X
    \subseteq \RR$ by
    \[
        X = \set{-2, 0, 1}
    \]
    and let $\rho(x, y) = \abs{x - y}$ be the restricted metric inherited from
    $\RR$. 

    We have that
    \[
        D_{1/2}(-2) = \set{-2},
    \]
    so $\set{-2}$ is closed. We also have that
    \[
        D_{1}(1) = \set{0, 1},
    \]
    so $\set{0, 1}$ is closed. But $\set{-2} = X \setminus \set{0, 1}$, so
    $\set{-2}$ is open. Hence, $D_{1/2}(-2)$ is open.
\end{proof}

Primary author: David Kraemer

\begin{minorEx}
    Find an open ball that is closed.
\end{minorEx}

\begin{proof}[Answer]
    Let $X$ be as defined in 4.21. We have that
    \[
        B_1(-2) = \set{-2},
    \]
    so $\set{-2}$ is open. We also have that 
    \[
        B_2(0) = \set{0, 1},
    \]
    so $\set{0, 1}$ is open. But $\set{-2} = X \setminus \set{0, 1}$, implying
    that $\set{-2}$ is closed. Hence, $B_1(-2)$ is closed.   
\end{proof}

Primary author: David Kraemer

\begin{minorEx}
    Prove that a sphere is closed.
\end{minorEx}

\begin{proof}
    Let $(X, \Omega)$ be a topology generated by the metric $\rho$. Let $x \in
    X$ and $r \in \RR_{++}$ be arbitrary, and consider $S_{r}(x)$. To show that
    $S_{r}(x)$ is closed, it suffices to show that $X \setminus S_{r}(x)$ is
    open. But
    \[
        X \setminus S_{r}(x) = B_{r}(x) \cup (X \setminus D_{r}(x)),
    \]
    following the defined constructions of spheres, balls, and disks. From the
    definition of the metric topology $\Omega$, we have that $B_{r}(x)$ is open.
    From 4.20, we have that $D_{r}(x)$ is closed; hence, $X \setminus D_{r}(x)$
    is closed. As the union of two open sets is open, we find that $X \setminus
    S_{r}(x)$ is open. Thus, $S_{r}(x)$ is closed.
\end{proof}

Primary author: David Kraemer

\begin{minorEx}
    Find a sphere that is open.
\end{minorEx}

\begin{proof}[Answer]
    Let $X$ be defined as in 4.21. We have that 
    \[
        S_1(0) = \set{1} = B_1(1),
    \]
    so $S_1(0)$ is open.
\end{proof}

Primary author: David Kraemer
