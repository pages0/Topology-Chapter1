\subsection{Operations with Metrics}
\begin{minorEx}
    \begin{enumerate}
        \item Prove that if $\rho_1$ and $\rho_2$ are two metrics in $X$, then
            $\rho_1 + \rho_2$ and $\max \set{\rho_1, \rho_2}$ are also metrics.
        \item Are the functions $\min\set{\rho_1, \rho_2}$, $\rho_1 \rho_2$, and
            $\rho_1 / \rho_2$ metrics? (By definition, for $\rho = \rho_1 /
            \rho_2$ we put $\rho(x, x) = 0$.)
    \end{enumerate}
\end{minorEx}

\begin{minorEx}
    Prove that if $\rho : X \times X \to \RR_+$ is a metric, then
    \begin{enumerate}
        \item the function $(x, y) \mapsto \frac{\rho(x,y)}{1 + \rho(x,y)}$ is a
            metric;
        \item the function $(x, y) \mapsto \min\set{p(x,y), 1}$ is a metric;
        \item the function $(x, y) \mapsto f(\rho(x,y))$ is a metric if $f$
            satisfies the following conditions:
            \begin{enumerate}
                \item $f(0) = 0$,
                \item $f$ is a monotone increasing function, and
                \item $f(x + y) \leq f(x) + f(y)$ for any $x, y \in \RR$.
            \end{enumerate}
    \end{enumerate}
\end{minorEx}

\begin{proof}
    \begin{enumerate}
        \item Let $d(x,y) = \frac{\rho(x,y)}{1 + \rho(x,y)}$. That $d(x,x) = 0$
            and $d(x, y) = d(y, x)$ follow from inspection of the construction
            of $d$. The triangle inequality emerges less elegantly, as we shall
            see:
            \begin{align*}
                \rho(x, y) &\leq \rho(x, z) + \rho(z, y) \\
                \rho(x, y) + \rho(x, y) (\rho(z, y) + \rho(x, z) (1 + \rho(y,
                z)))
                &\leq
                \rho(x, z) + \rho(z, y) + \rho(x, y) (\rho(z, y) + \rho(x, z) (1
                + 2\rho(y, z))) \\
                &\vdots
            \end{align*}
        \item Let $d(x,y) = \min\set{p(x,y), 1}$. That $d(x,x) = 0$ and $d(x,y)
            = d(y,x)$ follow from inspection. We show that the triangle
            inequality holds. 

            Suppose $\rho(x,y) \leq 1$. For any $z$, either $\rho(x,z) \leq 1$ and
            $\rho(z, y) \leq 1$, or not. If so, then
            \[
                \min{\set{\rho(x,z), 1}} = \rho(x,z) \qquad \text{and} \qquad
                \min{\set{\rho(z,y), 1}} = \rho(z,y),
            \]
            and applying the the triangle inequality gives the needed result.
            Otherwise, $\rho(x,y) \leq 1$, so the triangle inequality holds. 

            Suppose $\rho(x, y) \geq 1$. If both $\rho(x,z) \leq 1$ and
            $\rho(z,y) \leq 1$, we have
            \[
                d(x,y) = 1 \leq  \rho(x,y) \leq \rho(x,z) + \rho(z, y) = d(x,z)
                + d(y,z),
            \]
            verifying the triangle inequality. Otherwise, since $1 \leq 1$, the
            inequality holds easily.
        \item Let $d(x, y) = f(\rho(x,y))$, where $f$ is defined as above.
            Since $\rho(x, y) = 0$ if and only if $x = y$, and since $f(0) = 0$,
            we have $d(x, y) = 0$ if and only if $x = y$. Since $\rho(x, y) =
            \rho(y, z)$, we have $d(x,y) = f(\rho(x,y)) = f(\rho(y,z)) =
            d(y,z)$. Finally, since $\rho$ is nonnegative, and since $f$ is
            monotonically increasing, we have 
            \[
                d(x,y) = f(\rho(x, y)) \leq f(\rho(x,z) + \rho(z,y)).
            \]
            Since $f(a + b) \leq f(a) + f(b)$, we have
            \[
                f(\rho(x, z) + \rho(z, y)) \leq f(\rho(x,z) + \rho(z, y)).
            \]
            Hence,
            \[
                d(x,y) \leq d(x, z) + d(z, y),
            \]
            as needed.
    \end{enumerate}
\end{proof}

\begin{minorEx}
    Prove that the metrics $\rho$ and $\frac{\rho}{1 + \rho}$ are equivalent.
\end{minorEx}

\begin{proof}
    It suffices to show that the function $f : \RR_{+} \to \RR_+$ by
    $f(x) = \frac{x}{1 + x}$ is a bijection. If $d(x,y) =
    \frac{\rho(x, y)}{1 + \rho(x, y)}$, and $f$ is bijective, then for
    each $r \geq 0$ there exists a unique $r' \geq 0$ such that $B_r(x)$
    with respect to $\rho$ coincides with $B_{r'}(x)$ with respect to
    $d$. As such, the two topologies coincide.
    
    To see that $\frac{x}{1 + x}$ is injective, notice that if $x_1 <
    x_2$, then
    \[
        x_1 + x_1 x_2 < x_2 + x_1 x_2,
    \]
    so
    \[
        x_1(1 + x_2) < x_2 (1 + x_1),
    \]
    and hence
    \[
        f(x_1) = \frac{x_1}{1 + x_1} < \frac{x_2}{1 + x_2} = f(x_2).
    \]
    Thus, $f$ is injective. To see that $f$ is surjective, let $y \geq
    0$ be arbitrary. Take $x = \frac{y}{1 - y} \geq 0$, and we have
    \[
        f(x) = f\left( \frac{y}{1 - y} \right) =
        \frac{\frac{y}{1-y}}{1 + \frac{y}{1-y}} = y.
    \]
    Thus, $f$ is surjective, and hence $f$ is bijective.
\end{proof}
