\subsection{Metric Topology}

\begin{majorEx}%4.G
    The collection of all open balls in the metric space is a base for a certain
    topology.
\end{majorEx}

\begin{proof}
 Let $(X,\rho)$ be an arbitrary
 metric space. We will show that the set of all unions of open balls, and all
  finite intersections are a topology, and that $X$ and $\emptyset$
  are in the topology. We first see that as for any $a$, that $\rho(a,a)=0$,
  that any point $a\in X$ there is a ball $B_r(a)$ where $r>0$, 
  which contains it, and by taking the union of all such sets , we
  have that $X$ is in the topology.
  We also see that a ball with radius $0$ doesn't contain any points,
  and thus that the empty set is included in the topology. 

  We will now show that the topology is closed under finite
  intersection. We will call this intersection $A$. If this
  intersection is $\emptyset$, we know that it 
  is in the topology. Let $a$ be an arbitrary point in a finite intersection
  of sets in the topology. We see that if this intersection is $X$,
  that as we showed above, $a$ has an open ball contained within the
  set containing it. Otherwise, let $r= inf (\rho(a,b)| b \in X
  \setminus A)$. We see that if $a$ is contained in some open ball
  $B_r(c)$, then there would  be some $c\in \RR$ such that
  $\rho(c,a)<c<r$ by density of $\RR$, 
  and if any open ball is intersected with that open ball, that a
  similar property must hold if $a$ is contained within it. As
  intersection is the only way for a element of the topology to
  decrease in size, we know that $r>0$, and thus we know that $B_r(a)$
  contains only elements of $A$, by definition of a open ball, and
  $a\in B_r(a)$. As $a$ is in a open ball contained in $A$, and $a$ was arbitrary, we
  see that the topology is closed under finite intersection.

  We will now show that the topology is closed under arbitrary
  union. Let $A$ be a union of arbitrary sets of the topology, and let
  $a$ be an arbitary point in $A$. We see that by definition of union,
  $a$ is either in some open ball within, or in the finite intersection of
  open balls. Because we showed above that the topology is closed
  under finite intersection, we know that in both cases, that there is
  some open ball in $A$ which contains $a$. As $a$ is in a open ball 
  contained in $A$, and $a$ was arbitrary, we see that the topology is
  closed under arbitrary intersection.
  
  As we have proven all axioms of a topology, we know that the
  topology generated by set of all open balls is a topology.
\end{proof}

\begin{majorEx}%4.H
    Prove that the standard topological structure in $\RR$ introduced in Section
    2 is generated by the metric $(x, y) \mapsto \abs{x - y}$.
\end{majorEx}

\begin{proof}
  Let $(a,b)$ be an arbitrary open interval. We see that $\frac{a+b}{2}\in
  (a,b)$, and that $B_r(\frac{a+b}{2})$ where $r=\frac{a+b}{2}-a$ does
  not contain $a$ or $b$, but it does contain every point $x$ such
  that $|x-\frac{a+b}{2}|<\frac{a+b}{2}-a$, and thus 
  $-\frac{a+b}{2}+a<x-\frac{a+b}{2}<\frac{a+b}{2}-a$ which is
  equivalent to $a<x<b$. We thus have that $(a,b)$ is generated by the
  metric $(x, y) \mapsto \abs{x - y}$. As $(a,b)$ was arbitrary, we
  have that any open interval is generated by the metric $(x, y)
  \mapsto \abs{x - y}$, and thus the standard topological structure in
  $\RR$ introduced in Section 2 is generated by the metric $(x, y)
  \mapsto \abs{x - y}$.   
\end{proof}

\begin{minorEx}%4.19
    What topological structure is generated by the metric of $4.A$?
\end{minorEx}

\begin{proof}[Answer]
  Consider the ball $B_1(a)$ for an arbitrary point $a$. We
  see that if $x \in B_1(a)$, that $\rho(x,a)<1$, and by definition of
  the metric, we would thus have that $x=a$. We thus see as $a$ was
  arbitrary, that every element of the set has a set withing the
  topology containing only it. We see by the axioms of a topology,
  that this means that every union and every intersection of these
  elements is in the topology, and as any subset is made up of
  individual element of the set, we know that we can make any possible subset
  by taking the union of elements in the set. As any possible subset
  is in the topology, we thus know that $4.A$ generates the discrete topology.
\end{proof}

\begin{majorEx}%4.I
  A set $U$ is open in a metric space iff, together
  with each of its points, the set $U$ contains a
  ball centered at this point.
\end{majorEx}

\begin{proof}
  We will first prove that if $U$ is open in a metric space that it
  contains a ball centered at this point. Let $x \in U$ be arbitrary.
  By the definition the metric space, we know that $U$
  is the union of open balls. As $x \in U$, we know that for some open
  ball centered at $a$ with radius $r$,$x \in B_r(a)$. By definition
  of an open ball, we know that as $x \in B_r(a)$, we have
  $\rho(a,x)<r$, and thus by 4.E, we have that
  $B_{r-\rho(a,x)}(x)\subset B_r(a)$, and thus  the set $U$ contains a
  ball centered at $x$. As $x$ was arbitrary, we see that this is true
  for every point in $U$, and thus that if $U$ is open in a metric space that it
  contains a ball centered at this point.

  We will now show that if every point in $U$ contains a ball within $U$ centered
  at that point, that $U$ is an open set in a metric space. Consider
  the set formed by taking the union of all the balls centered around
  points in $U$. We would then see this union would contain every
  point in $U$, and that because each ball is within $U$, that this
  union would also be contained by $U$. We thus have that $U$ is equal
  to the union of all these balls, and thus that $U$ is an open set.
\end{proof}
