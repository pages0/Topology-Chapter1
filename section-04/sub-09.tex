\subsection{Metric Topology}

\begin{majorEx}%4.G
    The collection of all open balls in the metric space is a base for a certain
    topology.
\end{majorEx}

\begin{majorEx}%4.H
    Prove that the standard topological structure in $\RR$ introduced in Section
    2 is generated by the metric $(x, y) \mapsto \abs{x - y}$.
\end{majorEx}

\begin{minorEx}
    What topological structure is generated by the metric of 4.A?
\end{minorEx}

\begin{majorEx}%4.I
  A set $U$ is open in a metric space iff, together
  with each of its points, the set $U$ contains a
  ball centered at this point.
\end{majorEx}

\begin{proof}
  We will first prove that if $U$ is open in a metric space that it
  contains a ball centered at this point. Let $x \in U$ be arbitrary.
  By the definition the metric space, we know that $U$
  is the union of open balls. As $x \in U$, we know that for some open
  ball centered at $a$ with radius $r$,$x \in B_r(a)$. By definition
  of an open ball, we know that as $x \in B_r(a)$, we have
  $\rho(a,x)<r$, and thus by 4.E, we have that
  $B_{r-\rho(a,x)}(x)\subset B_r(a)$, and thus  the set $U$ contains a
  ball centered at $x$. As $x$ was arbitrary, we see that this is true
  for every point in $U$, and thus that if $U$ is open in a metric space that it
  contains a ball centered at this point.

  We will now show that if every point in $U$ contains a ball within $U$ centered
  at that point, that $U$ is an open set in a metric space. Consider
  the set formed by taking the union of all the balls centered around
  points in $U$. We would then see this union would contain every
  point in $U$, and that because each ball is within $U$, that this
  union would also be contained by $U$. We thus have that $U$ is equal
  to the union of all these balls, and thus that $U$ is an open set.
\end{proof}
