\subsection{Set-Theoretic Digression: De Morgan Formulas}
\begin{majorEx}
  Let $\Gamma$ ba an arbitrary collection of subsets of a set
  $X$. Then
  
\begin{align}
  X \setminus \bigcup_{A \in \Gamma} A = \bigcap_{A\in\Gamma} (X
  \setminus A) \\
  X \setminus \bigcap_{A \in \Gamma} A = \bigcup_{A\in\Gamma} (X
  \setminus A) 
\end{align}
\end{majorEx}

\begin{proof}
  We will first show that 
  $X \setminus \bigcup_{A \in \Gamma} A = \bigcap_{A\in\Gamma} (X
  \setminus A)$
  Let $x \in X \setminus \bigcup_{A \in\Gamma} A$ be arbitrary. We see
  by definition of union, that $x\in X$ and $x \notin A$ for any $A
  \in \Gamma$. We now can see that for any $A\in \Gamma$, $x \in (X \setminus A)$
  as $x\in X$, and  $x \notin A$. Because for any $A\in \Gamma$
  we know $x \in ( X \setminus A)$, we thus also know that 
  $x\in \bigcap_{A\in\Gamma} (X \setminus A)$, and as $x$ was
  arbitrary, we know that 
  $X \setminus \bigcup_{A \in \Gamma} A \subset \bigcap_{A\in\Gamma} (X
  \setminus A)$.

  Now let $y \in \bigcap_{A\in\Gamma} (X \setminus A)$ be
  arbitrary. We see that $y \in X$, and $y\notin A$ for each $A$. We
  can now see that if it were true that $y\in \bigcup_{A\in\Gamma}$,
  then for at least one $A\in \Gamma$, it would not be true that 
  $y \in X$, and $y\notin A$ for each $A$, and thus  
  $y\notin \bigcup_{A\in\Gamma}$. As it is also true that $y\in X$ we know
  that $y \in (X \setminus \bigcup_{A\in \Gamma})$, and as $y$ was
  arbitrary, we know that   
  $X \setminus \bigcup_{A \in \Gamma} A \supset \bigcap_{A\in\Gamma} 
  (X \setminus A)$. We can now see that by 1.E that 
  $X \setminus \bigcup_{A \in \Gamma} A  = \bigcap_{A\in\Gamma} (X
  \setminus A)$.

We will now show that 
$X \setminus \bigcap_{A \in \Gamma} A = \bigcup_{A\in\Gamma} (X
\setminus A)$.
Let $x \in X \setminus \bigcap_{A \in \Gamma} A$ be arbitrary. We see
by definition of set difference and intersection that $x \in X$, and
for some A $x\notin A$. We thus can see that for at least one $A$, $x\in
(X \setminus A)$, and thus that 
$x\in \bigcup_{A\in\Gamma} (X\setminus A)$. As $x$ was arbitrary, we
see that $X \setminus \bigcap_{A \in \Gamma} A \subset \bigcup_{A\in\Gamma} (X
\setminus A)$.

Now let $y \in \bigcup_{A\in\Gamma} (X\setminus A)$ be arbitrary. We
see that for some $A \in\Gamma$ $y \in X$ and $y\notin A$. Because
$y\notin A$ we know that $y\notin \bigcap_{A\in\Gamma} A$, as by
definition of intersection, if  $y\in \bigcap_{A\in\Gamma} A$ we would
have that $y\in A$. We now see that as $y\in X$, and $y \notin
\bigcap_{A\in\Gamma} A$ we know that $y \in X \setminus \bigcap_{A \in
  \Gamma} A$. As $y$ was arbitrary, we know that  
$X \setminus \bigcap_{A \in \Gamma} A \supset \bigcup_{A\in\Gamma} 
(X\setminus A)$. We now can see by 1.E that  
$X \setminus \bigcap_{A \in \Gamma} A = \bigcup_{A\in\Gamma}
(X\setminus A)$.
We thus know that 
$$X \setminus \bigcup_{A \in \Gamma} A = \bigcap_{A\in\Gamma} (X \setminus A) $$
and
$$X \setminus \bigcap_{A \in \Gamma} A = \bigcup_{A\in\Gamma} (X \setminus A) $$
\end{proof}