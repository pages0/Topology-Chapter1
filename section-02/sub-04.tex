\subsection{Additional Examples}

\begin{minorEx} % 2.4
  Let $X$ be $\RR$, and let $\Omega$ consist of the empty set and
  all infinte subsets of $\RR$. Is $\Omega$ a topological structure
\end{minorEx}

No, $\Omega$ is not a topology. Consider the sets $\{x | x \in [0, 1] or x = -3\}$ and $\{x | x \in [2, 3] or x = -3\}$. The intersection of these two sets is simply {3}, which does not belong to $\Omega$, violating the definition of a topology.

Primary author: Willie Kaufman

\begin{minorEx} % 2.5
  Let $X$ be $\RR$, and let $\Omega$ consist of the empty set and
  complements of all finite subsets of $\RR$. Is $\Omega$ a topological structure?
\end{minorEx}

\begin{proof}
We know that finite subsets of $\RR$ is a closed set, and the complement must be an open set. Since the union of any collection of open sets is still an open set that belongs to $\RR$, and the intersection of any finite collection of open sets is an open set that belongs to $\RR$. $\Omega$ also contains the empty set and $\RR$, so it is a topological structure on $X$.
\end{proof}

Primary Author: Jimin Tan

\begin{minorEx} % 2.6
    Let $(X, \Omega)$ be a topological space, $Y$ the set obtained from $X$ by
    adding a single element $a$. Is
    \[
        \Omega' = \set{\set{a} \cup U : U \in \Omega} \cup \set{\emptyset}
    \]
    a topological structure in $Y$?
\end{minorEx}

\begin{proof}[Answer]
    Yes. To see that Axiom 3 is satisfied, notice that since $X \in \Omega$,
    $\set{a} \cup X \in \Omega'$. We also have that $\emptyset \in \Omega'$.
    Thus, Axiom 3 holds.

    A brief aside on the relationship of $\Omega$ and $\Omega'$. Let $f :
    \Omega' \to \Omega$ be defined by $f(A) = A \setminus \set{a}$. We show that
    $f$ is a bijection. To see that $f$ is onto, let $U \in \Omega$ be
    arbitrary. Since $U \cup \set{a} \in \Omega'$ by definition, we have $f(U
    \cup \set{a}) = U$. Since $U$ was arbitrary, we have that $f$ is onto. To
    see that $f$ is one-to-one, consider the function $g : \Omega \to \Omega'$
    defined by $g(U) = U \cup \set{a}$, which is well-defined. We have
    \begin{align*}
        (f \circ g)(U) &= f(U\cup \set{a}) \\
                       &= U,
    \end{align*}
    while
    \begin{align*}
        (g \circ a)(A) &= g(A \setminus \set{a}) \\
                       &= A,
    \end{align*}
    showing that $g = \inv{f}$, which establishes that $f$ is one-to-one. Thus,
    $f$ is a bijection.

    To see that Axiom 1 is satisfied, let $S \subseteq \Omega'$ be arbitrary,
    and consider $\bigcup_{A \in S} A$.     
    \begin{align*}
        \bigcup_{A \in S} A &= \bigcup_{U \in f(S)} U \cup \set{a} \\
                            &= \set{a} \cup \left( \bigcup_{U \in f(S)} U
                            \right). 
    \end{align*}
    Since $\Omega$ is a topology, $\bigcup_{U \in f(S)} U = V$ for some $V \in
    \Omega$, so
    \[
        \bigcup_{A \in S} A = \set{a} \cup V,
    \]
    which is by definition an element of $\Omega'$. This verifies Axiom 1.

    To see that Axiom 2 is satisfied, let $S \subseteq \Omega'$ be an arbitrary
    finite set. Then
    \begin{align*}
        \bigcap_{A \in S} A &= \bigcap_{U \in f(S)} U \cup \set{a} \\
        &= \set{a} \cup \left( \bigcap_{U \in f(S)} U \right).
    \end{align*}
    Since $\Omega$ is a topology, there is a $V \in \Omega$ such that
    \[
        V = \bigcap_{U \in f(S)} U.
    \]
    Hence,
    \[
        \bigcap_{A \in S} A = \set{a} \cup V,
    \]
    which is by definition an element of $\Omega'$. This verifies Axiom 2.
  \end{proof}

  Primary author: David Kraemer

\begin{minorEx} % 2.7
  Is the set $\{ \emptyset, \{ 0 \}, \{0,1\} \}$ a topological
  structure in $\{ 0, 1\}$?
\end{minorEx}

\begin{proof}[Answer]
The set $\{ \emptyset, \{ 0 \}, \{0,1\} \}$ a topological
  structure in $\{ 0, 1\}$. We the union of any collection of elements
  in the topology will be either $\emptyset$, $\{0\}$ or $\{0,1\}$ and 
  thus Axiom 1 is satisfied. We similarly see that an intersection of
  any collection of elements in the set will result in an element that
  already exists in the topology, and thus Axiom 2 is
  satisfied. Finally, we see that as $\emptyset$ and $\{0,1\}$ are in
  the set, that Axiom 3 is satisfied.
\end{proof}

Primary Author: Reilly Noonan Grant

\begin{minorEx} % 2.8
  List all topological structures in a two-element set, say, in $\{0 , 1\}$
\end{minorEx}

$\{\emptyset, \{0, 1\}\}
\{\emptyset, \{0\}, \{1\}, \{0, 1\}\}
\{\emptyset, \{1\}, \{0, 1\}\}
\{\emptyset, \{0\}, \{0, 1\}\}$

are all topologies.








