\subsection{Simplest Examples}

\begin{majorEx} %2.A
    Check that the discrete topological space is a topological space, i.e., all
    axioms of topological structure hold true.
\end{majorEx}

\begin{proof}
  Let the discrete topological space be given by $(X, \Omega)$
  We will first show that Axiom (1) holds true. Let $A$ be the union
  of an arbitrary collection of sets in $\Omega$.  If $A = \Emptyset$, 
  we know by 1.C, and the definition of the discrete topological space
  that $A$ belongs to $\Omega$. Now suppose that $A$ is non 
  empty. Let $a\in A$  be arbitrary. We see that because $a\in A$,
  that by properties of a
  union, that $a$ is in at least one set in $\Omega$, and every set in
  $\Omega$ is a subset of $X$, we know that $a \in X$. As $a$ was
  arbitrary, we see that $A \subset X$, and thus $A$ belongs to
  $\Omega$. As $A$ was arbitrary, we know that Axiom (1) holds.
  

  We will now show that Axiom (2) holds true. Let $A$ be an arbitrary
  intersection of a finite collection of sets that are elements of
  $\Omega$. If $A = \Emptyset$, we know by 1.C that $A \subset X$ 
  and the definition of the discrete topological space that $A$ 
  belongs to $\Omega$. Now suppose that $A$ is non empty.
  Let  $a\in A$ be arbitrary. We see by the definition of
  intersection, and the definition of the discrete topological space
  that $a$ is an element of a subset of $X$, and thus $a \in X$. As $a
  \in X$, and $a$ was arbitrary, we see that $A \subset X$, and thus
  $A$ belongs to $\Omega$. As $A$ was arbitrary, we know that
  Axiom (2) holds.

  By 1.B and 1.C we see that $\emptyset$ and $X$ are subsets of $X$,
  and thus by the definition of the discrete topological space, we
  have that $\emptyset$ and $X$ belong to $\Omega$, and thus Axiom (3)
  holds true.

  As all Axioms of topological structure hold true, we know 
  that a discrete topological space is a topological space.
\end{proof}

Primary author: Reilly Noonan Grant

\begin{majorEx} %2.B
    The indiscrete topological space is a topological structure, is it not?
\end{majorEx}

\begin{proof}
    We show that the indiscrete topology $\Omega_I = \set{X, \emptyset}$ is
    indeed a topological structure. We see immediately that $X \in \Omega_I$ and
    $\emptyset \in \Omega_I$, so Axiom 3 is satisfied.

    To see that $\Omega_I$ is closed under arbitrary unions, let $\Omega_I'
    \subseteq \Omega_I$ be arbitrary. Then since $A \cup A = A$ and since 
    \[
        \bigcup_{A \in \Omega_I'} A
    \]
\end{proof}

\begin{minorEx} %2.1
    Let $X$ be the ray $[0, +\infty)$, and let $\Omega$ consist of $\emptyset$, %]
    $X$, and all rays $(a, +\infty)$ with $a \geq 0$. Prove that $\Omega$ is
    a topological space.
\end{minorEx}

\begin{minorEx} %2.2
    Let $X$ be a plane. Let $\Sigma$ consist of $\emptyset$, $X$, and all open
    disks centered at the origin. Is $\Sigma$ a topological structure?
\end{minorEx}

\begin{proof}[Answer---NEEDS FLESHING OUT]
    Yes. We see immediately that Axiom 3 is satisfied, since $\emptyset, X \in
    \Sigma$. In general, we will represent $D_r$ to indicate the open disk of
    radius $r > 0$ centered at the origin.

    To see that Axiom 1 holds, let $\Sigma' \subseteq \Sigma$ be arbitrary. To
    proceed, we observe that $D_r \subseteq D_{r'}$ whenever $r \leq r'$ (and
    equality holding exactly when $r = r'$). Let $f : \Sigma' \to \RR$ be
    defined by $f(D_r) = r$, and consider $f(\Sigma')$. Either $f(\Sigma')$ is
    bounded above, or it is unbounded. If it is bounded above, we have
    \[
        s = \sup f(\Sigma').
    \]
    In this case,
    \[
        \bigcup_{D_r \in \Sigma'} D_r = D_s \in \Sigma.
    \]
    Otherwise,
    \[
        \bigcup_{D_r \in \Sigma'} D_r = X \in \Sigma.
    \]
    In both cases, we see that the unions are elements of $\Sigma$.

    To see that Axiom 2 holds, let $\Sigma' \subseteq \Sigma$ be an arbitrary
    finite subset of $\Sigma$, and consider
    \[
        \bigcap_{D_r \in \Sigma'} D_r.
    \]
    Now, since $D_r \subseteq D_{r'}$ whenever $r \leq r'$, we have that $D_r
    \cap D_{r'} = D_r$ whenever $ r \leq r'$. In this case, $f(\Sigma')$ is
    finite, so by taking
    \[
        m = \min(f(\Sigma')),
    \]
    we have that
    \[
        \bigcap_{D_r \in \Sigma'} D_r = D_m.
    \]
\end{proof}

\begin{minorEx} %2.3
    Let $X$ consist of four elements: $X = \set{a, b, c, d}$. Which of the
    following collections of its subsets are topological structures in $X$,
    i.e., satisfy the axioms of topological structure:
    \begin{enumerate}
        \item $\emptyset, X, \set{a}, \set{b}, \set{a,c} \set{a,b,c},
            \set{a,b}$;
        \item $\emptyset, X, \set{a}, \set{b}, \set{a,b} \set{b,d}$;
        \item $\emptyset, X, \set{a,c,d}, \set{b,c,d}$?
    \end{enumerate}
\end{minorEx}

We see that (1) is a topological structure, however (2) and (3) both
fail to satisfy the axioms of topological structure. For (1), we see
that any union of the elements of the set is also included, and that
any intersection of the elements is also included. Because $\emptyset$
and $X$ are also in (1), we know that (1) satisfies the axioms of
topological struture, and thus is a topological structure. 

We see that $\{a,b\} \cup \{a,d\} = \{a,b,d\}$. As $\{a,b\}$ and
$\{a,d\} $ belong to (2), but $\{a,b,d\}$ does not we see that (2) does not
satisfy Axiom 1

We see that $\{a,c,d\} \cap \{b,c,d\} =\{c,d\}$. As $\{a,c,d\}$
and $\{b,c,d\}$ belong to (3), but $\{c,d\}$ does not, we know that 
(3) does not satisfy Axiom 2


Primary Author: Reilly Noonan Grant


