\subsection{Being Open or Closed}
\begin{majorEx} % 2.G
  Find Examples of sets that are 
  \begin{enumerate}
  \item both open and closed simultaneously
  \item neither open, nor closed
  \end{enumerate}
\end{majorEx}

\begin{minorEx} %2.10
\end{minorEx}

\begin{majorEx} %2.H
Is a closed segment $[a, b]$ closed in \mathds{R}?
Yes, it is. The complement of a closed segment $[a, b]$ is the union of the open sets $\{x | x < a\}$ and $\{x | x > b\}$, and so the closed segment is closed by the definition of a closed set and the standard topology on \mathds{R}.
\end{majorEx}

\begin{minorEx} %2.11
  Prove that the half-open interval $[0,1)$  is neither open nor
  closed, but is both a union of closed sets and an instersection of
  open sets.
\end{minorEx}
\begin{proof}
\end{proof}

\begin{minorEx} %2.12
Prove that the set $A = \{0\} \cup \{1/n | n \in N\}$ is closed in \mathds{R}.
\end{minorEx}

\begin{proof}
To show this, we will show that its complement is the arbitrary union of open sets. Note $A=\{0, 1/2, 1/3, 1/4, ...\}. Consider the collection of open intervals $\{A_i, A_i+1) | i \in \mathds{N}\}; call this collection $B$. This set contains no values in $A$, as they will only be endpoints of the open intervals. There is no way for a value in $A$ to be in an interval of $B$, because we're creating our collection such that every time we reach a value in $A$, we create a new interval in our collection. The union of all the sets in $B$, the open set $\{x | x < 0\}$ and the open set $\{x | x > 1\} is the complement of $A$ in \mathds{R}. To see this, choose an arbitrary element $r \in R$. If this element is in $A$, it must not be in $B$, as previously discussed, and the values in $A$ are bounded between 0 and 1, so we know the element being in $A$ precludes it from being in this union. The closure of this union is $\mathds{R}$, so if it is not in this union, it must be a limit point of $\mathds{R}$ that is not contained in the union. The only values that meet these criteria are exactly the values of $A$, so we know the element must be in $A$. We've shown that an element of \mathds{R} is in $A$ if and only if it is not in the union, so the union must be the complement. Then by the definition of closed, $A$ is closed.
