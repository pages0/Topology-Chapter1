\subsection{Being Open or Closed}
\begin{majorEx} % 2.G
  Find Examples of sets that are 
  \begin{enumerate}
  \item both open and closed simultaneously
  \item neither open, nor closed
  \end{enumerate}
\end{majorEx}

(1) Example: $\emptyset$ and $\RR$ are both open and closed at the same time in $\RR$.
\begin{proof}
They are all open since they belong to the topology on $\RR$, and we know that they are complements of each other, so they are both open and closed.
\end{proof}

(2) Example: $(0, 1]$ is neither closed or open on $\RR$
\begin{proof}
$(0, 1]$ is not open because we have to have infinite unions or intersections of opens to approach $1$. Then $(0, 1]^c = (-\infty, 0] \cup (1, +\infty)$ is still not open since we have to take infinite intersections or unions to approach $0$. Thus, this set is neither open nor closed.
\end{proof}

Primary Author: Jimin Tan

\begin{majorEx} %2.H
Is a closed segment $[a, b]$ closed in $\RR$?
\end{majorEx}

\begin{proof}
Yes, it is. The complement of a closed segment $[a, b]$ is the union of the open
sets $\{x | x < a\}$ and $\{x | x > b\}$, and so the closed segment is closed by
the definition of a closed set and the standard topology on $\RR$.
\end{proof}

\begin{minorEx}
    Give an explicit description of closed sets in
    \begin{enumerate}
        \item a discrete space;
        \item an indiscrete space;
        \item the arrow;
        \item the weird one;
        \item $\RR_{T_1}$.
    \end{enumerate}
\end{minorEx}

\begin{proof}[Answer]
    \begin{enumerate}
        \item Since $A \in \Omega$ whenever $A \subseteq X$, it follows that
            both $X \setminus A \in \Omega$ as well; hence, for all $A \subseteq
            X$, we have that $A$ is closed. The closed sets are therefore
            comprised of $\mathcal{P}(X) = \Omega$.
        \item Since $X = X \setminus \emptyset$ and $\emptyset = X \setminus X$,
            the closed sets are therefore $\set{X, \emptyset} = \Omega$.
        \item We have that
            \[
                \mathcal{F} = \set{[0, a] : a \geq 0} \cup \set{\RR_+,\emptyset}.
            \]
        \item We have that
            \[
                \mathcal{F} = \set{
                    \set{a,b,c,d}, \emptyset, \set{b,c,d}, \set{a,c,d},
                    \set{b,d}, \set{d}, \set{c, d}
                }.
            \]
        \item We have that 
            \[
                \mathcal{F} = \set{A : \abs{A} = n \text{ for some } n \geq 0}
                \cup \set{\RR}. \qedhere
            \]
    \end{enumerate}
\end{proof}

Primary author: David Kraemer


\begin{minorEx} %2.11
  Prove that the half-open interval $[0,1)$  is neither open nor
  closed, but is both a union of closed sets and an intersection of
  open sets.
\end{minorEx}
\begin{proof}
  We will first show that $[0,1)$ is not open. By the definition of an
  open set, we know that an open set is an arbitrary union of open
  sets. We now consider $0$. We see
  that as $0$ is in $[0,1)$, for $[0,1)$ to be open, there would have
  to be an open set which contains it, however, if any set $(a,b)$
  contains $0$, it would also contain the points less than $0$ and
  greater than $a$, and $[0,1)$ does not contains these points. We
  thus see that $[0,1)$ is not open. 
  
  We now will show that $[0,1)$ is
  not closed. Consider the complement of
  $[0,1)$, $(-\infty,0)\cup [1,\infty)$. We see that for this set to
  be open, some open set would have to contain $1$, however, if any
  open set $(a,b)$ contained $1$, then we would have that the points
  less than $1$ and greater than $0$ would be in the set, and we know
  that these points are not in the set, and thus $(-\infty,0)\cup
  [1,\infty)$ is not open, and its complement is $[0,1)$ not closed.

  We will now construct $[0,1)$ out of a union of closed
  sets. Consider a set of the form $[0,x]$ where $x\in (0,1)$. We see
  that its complement is $(-\infty,0) \cup (x,\infty)$ which is open,
  and thus $[0,x]$ is closed. We also see that if we take the union of
  every set of this form, that it will have all the same points as
  $[0,1)$. We will now prove this. Let $A$ be the union of every set of
  this form. Because for every $x$ $x<1$, we know that for any $x$ 
  $[0,x] \subset [0,1)$, and thus for any $a \in A$, as $a\in [0,x]$
  for some $x$, $a\in [0,1)$, and thus $A\subset [0,1)$. Now, let 
  $b \in [0,1)$ be arbitrary. We see that if $b=0$, that $b\in A$, and
  that if $b\neq 0$, that $b\in (0,1)$, and thus $[0,b]$ is a set in
  $A$, and thus $b\in A$. As $b$ was arbitrary $A\supset [0,1)$. We
  now see that by 1.E that $A = [0,1)$.

  We will now construct $[0,1)$ out of an intersection of open
  sets. Consider a set of the form $(-\frac{1}{n}, 1)$ where $n\in
  \NN^{>0}$. Because both $-\frac{1}{n}$ and $1$ are real numbers, we know
  that $(-\frac{1}{n}, 1)$ is an open set. We will call the
  intersection of all sets of this form $B$. Let $a \in [0,1)$ be
  arbitrary. We see that for any $n$, $a\geq 0> -\frac{1}{n}$, and
  $a<1$, and thus $a\in B$, and $[0,1)\subset B$. Now let $b\in B$ be
  arbitrary. Suppose $b <0$. We would then see that for some $n$, 
  $b>-\frac{1}{n}>0$, and thus $b$ would not be in $B$. We thus can
  see that $b\geq 0$. We also see that as $b\in (-1,1)$, that $b<1$.
  We thus see that as $b\geq 0$ and $b<1$, that $b \in [0,1)$, and
  thus as $b$ was arbitrary, $B\subset [0,1)$. We now can see that by
  1.E $B=[0,1)$
\end{proof}

Primary author: Reilly Noonan Grant

\begin{minorEx} %2.12
Prove that the set $A = \{0\} \cup \{1/n | n \in N\}$ is closed in $\RR$.
\end{minorEx}

\begin{proof}
To show this, we will show that its complement is the arbitrary union
of open sets. Note $A=\{0, 1/2, 1/3, 1/4, ...\}$. Consider the
collection of open intervals $\{A_i, A_i+1) | i \in \NN \}$; call this
collection $B$. This set contains no values in $A$, as they will only
be endpoints of the open intervals. There is no way for a value in $A$
to be in an interval of $B$, because we're creating our collection
such that every time we reach a value in $A$, we create a new interval
in our collection. The union of all the sets in $B$, the open set $\{x
| x < 0\}$ and the open set $\{x | x > 1\}$ is the complement of $A$ in
$\RR$. To see this, choose an arbitrary element $r \in R$. If this element is in $A$, it must not be in $B$, as previously discussed, and the values in $A$ are bounded between 0 and 1, so we know the element being in $A$ precludes it from being in this union. The closure of this union is $\RR$, so if it is not in this union, it must be a limit point of $\RR$ that is not contained in the union. The only values that meet these criteria are exactly the values of $A$, so we know the element must be in $A$. We've shown that an element of $\RR$ is in $A$ if and only if it is not in the union, so the union must be the complement. Then by the definition of closed, $A$ is closed.
\end{proof}

Primary author: Willie Kaufman
