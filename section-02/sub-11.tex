\subsection{Open Sets on Line}
\begin{majorEx}
  Prove that every open subset of the real line is a union of disjoint
  open intervals.
\end{majorEx}

\begin{proof}
  We will show this by using proof by contradiction. We first notice
  that by the definiton of the canonical topological space of the real
  line, that open sets are arbitary unions of open intervals. We assume for
  sake of contradiction that there exists an open subset $X$ of the real
  line where it is neccesary for some collection of open intervals to intersect.
  We will let $A$ be an arbitrary one of these intersections. Let
  $c\in A$ be arbitrary, and let $B$ be the open interval $(a,b)$
  where $a$ is chosen to be the real number closest to $c$ which satisfies
  $a<c$, and $a\notin X$, or $-\infty$ if there is no such element,
  and where $b$ is chosed to be the real number closest to $c$ which
  satisfies $c<b$ and $b\notin X$, or $+\infty$ if there is no such
  element. As $a$ and $b$ are the closest elements to $c$ which are
  not in $X$, we know that $(a,b)\subset X$. We now can see that if
  $A\subset (a,b)$ that every element in $A$ could have been covered
  by one open set, and thus $A$ would not have been neccessary, and
  every element of $X$ in $A$ could have been covered by one open
  interval, and we would have a contradiction. We now see that as $A$
  is the intersection of arbitrary unions of open sets, that $A$ is an
  open interval, and thus has the form $(\alpha,\beta)$ from some
  $\alpha,\beta \in \RR$. We can see that if $\alpha< a$, that $a\in
  A$, and as $a\notin X$ we would see that $A$ is not a subset of $X$,
  and thus $\alpha \geq a$. We similarly see that $\beta \leq b$. We
  now pick an arbitrary element $d$ from $A$. As $d\in A$, we know
  $\alpha < d < \beta$, and thus $a<d<b$. We thus see that $d\in
  (a,b)$, and as $d$ was arbitrary, we have that $A\subset (a,b)$  and
  thus have a contradiction, and know that there is no open subset of
  the real line whiere it is neccesary for some collection of open
  intervals to intersect, and thus  every open subset of the real line
  can be described as a union of disjoint
  open intervals.

\end{proof}
