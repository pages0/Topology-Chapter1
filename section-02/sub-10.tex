\subsection{Neighborhoods}
2.15 Give an explicit description of all neighborhoods of a point in
\begin{enumerate}
\item a discrete space;
\item an indiscrete space
\item The arrow;
\item X = $\{a,b,c,d\}$
\item a connected pair of points;
\item particular point topology
\end{enumerate}

\begin{enumerate}
  \item A neighborhood of a point in a discrete space topology $(X,\Omega)$ would
    be all subsets of $X$ which contain that point.
  \item A neighborhood of a point in a indiscret space topology
    $(X,\Omega)$ would just be $X$, as $X$and $\emptyset$ are the only
    elements of $\Omega$, and all of the points are in $X$.
  \item Let $(a,+\infty)$ be an element of the arrow. We see that a
    neighborhood of this point would be $[0,+\infty)$ or any ray of the form
    $(b,+\infty)$ where $b\leq a$, as otherwise, $a$ would not be
    included, in the set, and thus it would not be a neighborhood
  \item We see that if $X = \{a,b,c,d\}$, that the neighborhoods of
    $a$ are
    $\{a\}$,$\{a,b\}$,$\{a,c\}$,$\{a,d\}$$\{a,b,c\}$,$\{a,b,d\}$
    ,$\{a,c,d\}$  and$\{a,b,c,d\}$, and we can see that for any other
    point, we can get their neighborhoods by swaping all occurances of
    $a$ with that point, and that point with $a$
  \item The neighborhood of a point in a connected pair of points
    topology would just be the set of the point by itself, and the set
    of the point with its pair, as no other set includes that point.
  \item The neighborhood of a point in a particular point topology
    would depend on the point in question. If that point is not the
    particular point, then it would be similar to the discrete
    topology, except that every set would also include the particular
    point, and if the point is the particular point, then every set in
    the topology would be a neighborhood.
\end{enumerate}

Primary Author: Reilly Noonan Grant