\subsection{Inclusion Versus Belonging}%1'9
            \begin{majorEx}%1.F
            $x \in A$ if and only if $\{x\} \subset A$.
            \end{majorEx}

\begin{proof} We will first show that if $\{x\} \subset A$, then $x \in A$. We can see that as $\{x\}$ is a set described by listing all of its elements, that $x \in \{x\}$. We also see that
as $\{x\} \subset A$, that all of the elements of $\{x\}$ are also elements of $A$, and thus $x \in A$. We thus know that if $\{x\} \subset A$, then $x \in A$.

We will now show that if $x \in A$, then $\{x\} \subset A$. We can see that as $\{x\}$ is a set described by listing all of its elements, that $x$ is the only element in $\{x\}$.
We also see that as $x \in A$, that all elements of $\{ x\}$ belong to $A$, and thus $\{x\} \subset A$. We now can see that $x \in A$ if and only if $\{x\} \subset A$.
\end{proof}
Primary author: Reilly Noonan Grant			
            \begin{majorEx}%1.G
            \textbf{\textit{Non-Reflexivity of Belonging.}} Construct a set $A$ such that $A \not\in A$.
            The example, $\{1\} \not\in \{1\}$ shows the statement above. The set that contains $\{1\}$ is $\{\{1\}\}$.
                    Primary author: Jimin Tan            
                    \end{majorEx}
            
\begin{majorEx}%1.H
            \textbf{\textit{Non-Transitivity of Belonging.}} Construct three sets A, B, and C such that $A \in B$ and $B \in C$, but $A \not \in C$.
            \newline $A = \{1\}$
            \newline $B = \{\{1\}, 2\}$
            \newline $C = \{\{\{1\}, 2\}, 3\}$
            \newline
            \end{majorEx}
            
Primary author: Willie Kaufman
