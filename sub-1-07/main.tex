\subsection{Properties of Inclusion}%1'7
				\begin{majorEx}[Reflexivity of Inclusion]%1.B
                Any set includes itself: $A \subset A$ holds true for any $A$.
				\end{majorEx}
                \begin{proof}
                  We first let $A$ be an arbitrary Set. Suppose that $A$ is the empty set. We would then see that as $A$ doesn't have any elements, that it is vacuously true that every element of $A$ is in $A$, and thus  $A \subset A$. Now suppose that $A$ is non empty. Let $a \in A$ be arbitrary. We see that $a \in A$, and as $a$ was arbitrary, we know that this is true for all elements of $A$, and thus $A \subset A$. As $A$ was arbitrary, and $A \subset A$ for all cases, we see that $A \subset A$ holds true for any $A$.
                \end{proof}
                Primary author: Reilly Noonan Grant
				\begin{majorEx}%1.C
					\textbf{\textit{The Empty Set Is Everywhere.}} The inclusion $\emptyset \subset A$ holds true for any A. In other words, the empty set is present in each set as a subset.
                    \begin{proof}
                    Assume that $\emptyset \subset A$ does not hold, by defintion of inclusion, there exist at least one element $a \in \emptyset$ such that $a \not\in A$. Since $\emptyset$ does not contain any element, we have a contradiction.
                    \end{proof}
				\end{majorEx}
				\begin{majorEx}%1.D
					If A , $B$, and $C$ are sets, $A \subset B$, and $B \subset C$, then $A \subset C$. 
				\end{majorEx}
                \begin{proof}
                Let $A$, $B$, and $C$ be arbitrary sets such that $A \subset B$ and $B \subset C$. If $A$ is the empty set, it is true that $A \subset C$. If $A$ is nonempty, choose an arbitrary element $a \in A$. Because $A \subset B$, we know that $a \in B$, and similarly because  $B \subset C$, $a \in C$. Since $a$ was arbitrary, $A \subset C$. \newline
                \end{proof}
      		 Primary author: Willie Kaufman
